%% Generated by Sphinx.
\def\sphinxdocclass{report}
\documentclass[letterpaper,10pt,english]{sphinxmanual}
\ifdefined\pdfpxdimen
   \let\sphinxpxdimen\pdfpxdimen\else\newdimen\sphinxpxdimen
\fi \sphinxpxdimen=.75bp\relax

\PassOptionsToPackage{warn}{textcomp}
\usepackage[utf8]{inputenc}
\ifdefined\DeclareUnicodeCharacter
% support both utf8 and utf8x syntaxes
  \ifdefined\DeclareUnicodeCharacterAsOptional
    \def\sphinxDUC#1{\DeclareUnicodeCharacter{"#1}}
  \else
    \let\sphinxDUC\DeclareUnicodeCharacter
  \fi
  \sphinxDUC{00A0}{\nobreakspace}
  \sphinxDUC{2500}{\sphinxunichar{2500}}
  \sphinxDUC{2502}{\sphinxunichar{2502}}
  \sphinxDUC{2514}{\sphinxunichar{2514}}
  \sphinxDUC{251C}{\sphinxunichar{251C}}
  \sphinxDUC{2572}{\textbackslash}
\fi
\usepackage{cmap}
\usepackage[T1]{fontenc}
\usepackage{amsmath,amssymb,amstext}
\usepackage{babel}



\usepackage{times}
\expandafter\ifx\csname T@LGR\endcsname\relax
\else
% LGR was declared as font encoding
  \substitutefont{LGR}{\rmdefault}{cmr}
  \substitutefont{LGR}{\sfdefault}{cmss}
  \substitutefont{LGR}{\ttdefault}{cmtt}
\fi
\expandafter\ifx\csname T@X2\endcsname\relax
  \expandafter\ifx\csname T@T2A\endcsname\relax
  \else
  % T2A was declared as font encoding
    \substitutefont{T2A}{\rmdefault}{cmr}
    \substitutefont{T2A}{\sfdefault}{cmss}
    \substitutefont{T2A}{\ttdefault}{cmtt}
  \fi
\else
% X2 was declared as font encoding
  \substitutefont{X2}{\rmdefault}{cmr}
  \substitutefont{X2}{\sfdefault}{cmss}
  \substitutefont{X2}{\ttdefault}{cmtt}
\fi


\usepackage[Bjarne]{fncychap}
\usepackage{sphinx}

\fvset{fontsize=\small}
\usepackage{geometry}


% Include hyperref last.
\usepackage{hyperref}
% Fix anchor placement for figures with captions.
\usepackage{hypcap}% it must be loaded after hyperref.
% Set up styles of URL: it should be placed after hyperref.
\urlstyle{same}
\addto\captionsenglish{\renewcommand{\contentsname}{Contents:}}

\usepackage{sphinxmessages}
\setcounter{tocdepth}{1}



\title{s2Dcd}
\date{Jan 21, 2020}
\release{0.1.0}
\author{Alessandro Comunian}
\newcommand{\sphinxlogo}{\vbox{}}
\renewcommand{\releasename}{Release}
\makeindex
\begin{document}

\pagestyle{empty}
\sphinxmaketitle
\pagestyle{plain}
\sphinxtableofcontents
\pagestyle{normal}
\phantomsection\label{\detokenize{index::doc}}
\begin{figure}[htbp]
\centering

\noindent\sphinxincludegraphics[width=200\sphinxpxdimen]{{logo_thom}.png}
\end{figure}




\chapter{Code purpose}
\label{\detokenize{purpose:code-purpose}}\label{\detokenize{purpose::doc}}
\sphinxcode{\sphinxupquote{s2Dcd}} :
\begin{quote}

\sphinxcode{\sphinxupquote{s}}   sequential

\sphinxcode{\sphinxupquote{2D}}  bi\sphinxhyphen{}dimensional (multiple\sphinxhyphen{}point statistics simulation with)

\sphinxcode{\sphinxupquote{cd}} conditioning data
\end{quote}

The code \sphinxtitleref{s2Dcd} allows to apply the sequential 2D (multiple\sphinxhyphen{}point
simulation) with conditioning data approach described in the paper by
\sphinxcite{purpose:comunian2012}. For a list of publications where the \sphinxtitleref{s2Dcd} was
used, please see the section \sphinxhref{publications.html}{publications}.

\begin{sphinxadmonition}{note}{Note:}
At the moment the \sphinxcode{\sphinxupquote{s2Dcd}} is it only a \sphinxstylestrong{wrapper}
library: it requires an external MPS simulation engine to work.
\end{sphinxadmonition}

\begin{sphinxadmonition}{warning}{Warning:}
Since 2017 only the direct sampling (DS) version of the
\sphinxcode{\sphinxupquote{s2Dcd}} is supported. Therefore, in the following
documentation all references to the \sphinxcode{\sphinxupquote{Impala}} MPS
simulation engine should be considered outdated.
\end{sphinxadmonition}


\section{Not only \sphinxstyleliteralintitle{\sphinxupquote{s2Dcd}}…}
\label{\detokenize{purpose:not-only-s2dcd}}
The package can be used to do other things in addition to the
application of the \sphinxtitleref{s2Dcd} approach. For example, it contains some
modules which are used as interface to the multiple\sphinxhyphen{}point simulation
(MPS) codes \sphinxtitleref{DeeSse} and \sphinxtitleref{Impala}. Therefore, if for example you need
to run a number of simulation tests with different parameters, you can
use \sphinxtitleref{Python} and the interface to the two codes to create customized
benchmarks with the flexibility provided by \sphinxtitleref{Python}. See the
documentation provided in the appendices and the examples for more
details.

Also, some simple functions that allows to read and write from \sphinxtitleref{numpy}
to \sphinxtitleref{GSLIB} and \sphinxtitleref{VTK} are provided.

If you have any questions, you want to contribute, suggest new
features, you’ve found a bug…  you can write me an e\sphinxhyphen{}mail:
\sphinxtitleref{alessandro DOT comunian AT gmail DOT com}


\section{MPS simulation engines}
\label{\detokenize{purpose:mps-simulation-engines}}
As already mentioned, \sphinxtitleref{s2Dcd} is only a wrapper library and requires
an MPS simulation engine to work.  This version contains and interface
to the DeeSse MPS simulation engine (see \sphinxhref{http://www.randlab.org/research/deesse/}{this link} for mode info.)  Another
simulation engine (\sphinxtitleref{Impala}) was supported until 2017, but the current
development of the interface to this engine is discontinued.

Nevertheless, users are encouraged to develop interfaces to their
favorite MPS simulation engine (like for example \sphinxhref{http://sgems.sourceforge.net/}{SGeMS} or \sphinxhref{https://github.com/ergosimulation/mpslib}{MPSLib}).


\section{What’s new in this version}
\label{\detokenize{purpose:what-s-new-in-this-version}}\begin{itemize}
\item {} 
Removed the dependencies from the \sphinxtitleref{VTK} libraries.

\item {} 
Removed the dependencies from some additional modules derived by
\sphinxtitleref{Fortran90} subroutines.

\item {} 
Solved some bugs.

\item {} 
Added some new examples.

\end{itemize}


\section{If you use \sphinxstyleliteralintitle{\sphinxupquote{s2Dcd}}}
\label{\detokenize{purpose:if-you-use-s2dcd}}
Please:
\begin{itemize}
\item {} 
Let us know!

\item {} 
Cite the paper \sphinxcite{purpose:comunian2012}

\end{itemize}


\section{Acknowledgments}
\label{\detokenize{purpose:acknowledgments}}
Many thanks to Philippe Renard and Julien Straubhaar for their support
and suggestions, to Andrea Borghi for the fruitful discussions and for
finding some bugs, and to my brother Thomas for the design of the
s2Dcd logo.


\chapter{Installation}
\label{\detokenize{installation:installation}}\label{\detokenize{installation::doc}}
All the modules included in the “package” are written in Python
(version 3.X), which is platform independent. Therefore, the “package”
should work on Linux, Mac OS X and Windows. Some tests were performed
on all the three platforms, but at the moment the MPS engines
implemented in the \sphinxtitleref{s2Dcd} are available only for Linux and Windows.


\section{Requirements}
\label{\detokenize{installation:requirements}}\begin{itemize}
\item {} 
\sphinxhref{http://python.org}{Python} (tested version 3.6.9 on Linux). Older versions should also work.

\item {} 
If you are on Linux, you will also need the python development packages.

\item {} 
The Python numerical library \sphinxhref{http://numpy.scipy.org/}{numpy} and
\sphinxhref{http://pandas.pydata.org/}{pandas} and the Python package installer \sphinxcode{\sphinxupquote{pip}}.

\item {} 
The Multiple\sphinxhyphen{}point statistics simulation engines \sphinxtitleref{DeeSse} and/or \sphinxtitleref{Impala}.

\end{itemize}

\begin{sphinxShadowBox}
\sphinxstylesidebartitle{Suggested software (optional)}

Some software that you might find useful:
\begin{itemize}
\item {} 
\sphinxhref{http://www.paraview.org/}{ParaView}, useful to visualize the
\sphinxtitleref{VTK} output of \sphinxtitleref{Impala}.

\item {} 
\sphinxhref{http://sgems.sourceforge.net/}{SGeMS}, useful to visualize the
\sphinxtitleref{GSLIB} output of \sphinxtitleref{DeeSse}.

\end{itemize}
\end{sphinxShadowBox}


\subsection{Python, numpy and pandas installation}
\label{\detokenize{installation:python-numpy-and-pandas-installation}}

\subsubsection{The “easy” way}
\label{\detokenize{installation:the-easy-way}}
If you do not want to install all the Python packages separately,
there are some bundled distributions that allow to install all the
required libraries in “one click”. One of this distribution is
\sphinxhref{https://www.continuum.io/why-anaconda}{Anaconda}, which is
available for MS Windows, Linux and Mac OS X (note however that this
will require 200/300 Mbytes on your hard drive). Alternatively, you
can use another “bundled distribution” or install each python package
separately as explained in the following sections.

\begin{sphinxadmonition}{warning}{Warning:}
When you download Anaconda, select the Python \sphinxstyleemphasis{3.X}
version and \sphinxstylestrong{NOT} the one for Python \sphinxstyleemphasis{2.X}.
\end{sphinxadmonition}

\begin{sphinxadmonition}{note}{Note:}
Personally, when working on Linux, I prefer to install Python packages
using the OS package manager.
\end{sphinxadmonition}


\subsubsection{MS Windows}
\label{\detokenize{installation:ms-windows}}
If you are on MS Windows, I strongly suggest you to use something like
the aforementioned bundled distribution \sphinxtitleref{Anaconda}.


\subsubsection{Linux}
\label{\detokenize{installation:linux}}
Use you package manager, or \sphinxcode{\sphinxupquote{sudo apt\sphinxhyphen{}get install}} to install
\sphinxcode{\sphinxupquote{python3}}, \sphinxcode{\sphinxupquote{python3\sphinxhyphen{}dev}}, \sphinxcode{\sphinxupquote{python3\sphinxhyphen{}numpy}} and
\sphinxcode{\sphinxupquote{python3\sphinxhyphen{}pandas}}.  If \sphinxtitleref{pandas} is not available for your
distribution, please check on the last installation instructions \sphinxhref{http://pandas.pydata.org/}{here}.


\subsubsection{Mac OS X}
\label{\detokenize{installation:mac-os-x}}
Python and numpy are available for Mac OS X too. However, at the
moment the binaries of the MPS simulation engines with an interface to
the \sphinxtitleref{s2Dcd} are not available for Mac OS X. You can still use some
utility scripts to work with \sphinxtitleref{GSLIB} files.


\subsection{The multiple\sphinxhyphen{}point simulation engine \sphinxtitleref{DeeSse}}
\label{\detokenize{installation:the-multiple-point-simulation-engine-deesse}}
Actually, the python module \sphinxtitleref{s2Dcd} is only a wrapper for a MPS
simulation engine. For the moment, the implementation allows to use the
MPS simulation engine \sphinxtitleref{DeeSse}.

The executable of your MPS simulation engine (in the case of the
\sphinxtitleref{DeeSse} they are called \sphinxcode{\sphinxupquote{deesse}} and \sphinxcode{\sphinxupquote{deesseOMP}}; add \sphinxcode{\sphinxupquote{.exe}} if
you are working on MS Windows), should be located in your \sphinxcode{\sphinxupquote{bin}}
directory, in the working directory or in a directory listed in the
\sphinxcode{\sphinxupquote{PATH}} environment variable.

\begin{sphinxShadowBox}
\sphinxstylesidebartitle{Adapt the names of the MPS binaries}

If the provided binaries have a different name…

The name of the binary of the simulation code can be changed
using the variable \sphinxtitleref{s2Dcd.mps\_exes}, which is a Python dictionary.
\end{sphinxShadowBox}

To add a directory (i.e. \sphinxcode{\sphinxupquote{C:\textbackslash{}Users\textbackslash{}alex\textbackslash{}my\_dir}} for MS Windows, or \sphinxcode{\sphinxupquote{/home/alex/my\_dir}})
to your path variable do:


\subsubsection{MS Windows}
\label{\detokenize{installation:id1}}\begin{enumerate}
\sphinxsetlistlabels{\arabic}{enumi}{enumii}{}{.}%
\item {} 
\sphinxtitleref{Start \textgreater{} Control Panel \textgreater{} System and Security \textgreater{} Advances System Settings}

\item {} 
Click on the button \sphinxtitleref{Environment variables} at the bottom right.

\item {} 
Under \sphinxtitleref{System Variables}, select the line containing the variable \sphinxtitleref{Path}.

\item {} 
Edit the variable: add to the end of the variable value  \sphinxcode{\sphinxupquote{;C:\textbackslash{}Users\textbackslash{}alex\textbackslash{}my\_dir\textbackslash{}}}
(note the comma \sphinxstylestrong{;})

\end{enumerate}


\subsubsection{Linux}
\label{\detokenize{installation:id2}}
Add the following line to your \sphinxtitleref{.bash\_aliases} file (eventually, create it and make
sure that your \sphinxtitleref{.bashrc} loads it):

\begin{sphinxVerbatim}[commandchars=\\\{\}]
export PATH=\PYGZdl{}\PYGZob{}PATH\PYGZcb{}:/home/alex/my\PYGZus{}dir
\end{sphinxVerbatim}

\begin{sphinxadmonition}{note}{Note:}
Before testing the \sphinxtitleref{s2Dcd} module, check if the MPS simulation
engines work correctly.
\end{sphinxadmonition}


\section{Installation of the module \sphinxtitleref{s2Dcd} and other python tools}
\label{\detokenize{installation:installation-of-the-module-s2dcd-and-other-python-tools}}
Download the sources from Github. From the command line (use the Anaconda prompt if you installed it), move into the directory
\sphinxcode{\sphinxupquote{s2Dcd}} (if needed, unpack it)  and type the command:

\begin{sphinxVerbatim}[commandchars=\\\{\}]
\PYG{n}{pip} \PYG{n}{install} \PYG{o}{\PYGZhy{}}\PYG{n}{e} \PYG{o}{.}
\end{sphinxVerbatim}

To check if the installation was successful, you can run the python shell and try import the
main module:

\begin{sphinxVerbatim}[commandchars=\\\{\}]
\PYG{g+gp}{\PYGZgt{}\PYGZgt{}\PYGZgt{} }\PYG{k+kn}{import} \PYG{n+nn}{s2Dcd}\PYG{n+nn}{.}\PYG{n+nn}{s2Dcd}
\end{sphinxVerbatim}


\section{Some details about the main modules:}
\label{\detokenize{installation:some-details-about-the-main-modules}}\begin{description}
\item[{\sphinxtitleref{s2Dcd.py}}] \leavevmode
the main module containing for the functions for the \sphinxtitleref{s2Dcd} simulations.

\item[{\sphinxtitleref{deesse.py}}] \leavevmode
a simple interface to the parameters required by \sphinxtitleref{DeeSse}.

\item[{\sphinxtitleref{gslibnumpy.py}}] \leavevmode
to convert from numpy and GSLIB and vice versa.

\item[{\sphinxtitleref{utili.py}}] \leavevmode
some simple utilities…

\end{description}

More details about these in section {\hyperref[\detokenize{appendices:appendices}]{\sphinxcrossref{\DUrole{std,std-ref}{Appendices}}}}.


\chapter{Examples}
\label{\detokenize{examples:examples}}\label{\detokenize{examples::doc}}
Some usage examples of the \sphinxcode{\sphinxupquote{s2Dcd}} module, including a \sphinxtitleref{commented}
example.

\begin{sphinxadmonition}{warning}{Warning:}
Some of the examples contained in the provided directory
are very work in progress. If the directory related to an
example contains a file \sphinxcode{\sphinxupquote{README.txt}} where it is
clearly stated that the example is \sphinxstylestrong{work in progress},
do not trust too much the corresponding example.
\end{sphinxadmonition}


\section{How to run the examples}
\label{\detokenize{examples:how-to-run-the-examples}}

\subsection{MS Windows}
\label{\detokenize{examples:ms-windows}}
\begin{sphinxadmonition}{note}{Note:}
Here we suppose that you are using the default GUI interface
(‘IDLE’) provided with Python. However, if you installed
Python using \sphinxtitleref{Anaconda}, there is a nicer user friendly GUI
named \sphinxtitleref{spyderlib} that you can use instead.
\end{sphinxadmonition}

If you have installed \sphinxtitleref{Python}, then you should have also the Python
\sphinxtitleref{IDLE} (a GUI interface). From the interface you can open one of the
files in the \sphinxcode{\sphinxupquote{examples}} directory, and from the menu of the \sphinxtitleref{IDLE}
select \sphinxtitleref{run}. Another window should open and show you the advancement
of the simulation.

Alternatively, you can use the \sphinxtitleref{Command Prompt}, like this:

\begin{sphinxVerbatim}[commandchars=\\\{\}]
\PYG{n}{C}\PYG{p}{:}\PYGZbs{}\PYG{n}{Users}\PYGZbs{}\PYG{n}{toto}\PYGZbs{}\PYG{n}{examples}\PYGZbs{}\PYG{l+m+mi}{01}\PYG{n}{\PYGZus{}Strebelle}\PYG{o}{\PYGZgt{}} \PYG{n}{python}\PYG{o}{.}\PYG{n}{exe}  \PYG{n}{s2Dcd\PYGZus{}run}\PYG{o}{\PYGZhy{}}\PYG{n}{ex01}\PYG{o}{.}\PYG{n}{py}
\end{sphinxVerbatim}

\begin{sphinxadmonition}{note}{Note:}
If you are using \sphinxtitleref{spyderlib} to run
your scripts (and you don’t have any Python installation
other than the one that comes with the \sphinxtitleref{Anaconda}), it
can be useful, once you set the \sphinxcode{\sphinxupquote{PYTHONPATH}} environment
variable from the \sphinxtitleref{spyderlib} menu (inside the \sphinxtitleref{Python Path
Manager} menu), to push the button \sphinxtitleref{synchronize…} (and
accept with \sphinxtitleref{Yes}). This should allow you to use \sphinxtitleref{s2Dcd} also
with an external python shell. Then, to run the \sphinxtitleref{s2Dcd}, go
to \sphinxtitleref{Run \textgreater{} Configuration per file…} and select \sphinxtitleref{Execute in an
external system terminal}, and tick \sphinxtitleref{Interact with the python
console after execution}. Then you should able to run the
\sphinxtitleref{s2Dcd} with these option without seeing plenty of
command\sphinxhyphen{}prompt windows popping up…
Note that you will probably need to restart the Python kernel.
\end{sphinxadmonition}


\subsection{Linux}
\label{\detokenize{examples:linux}}
In the directory of the each example, you can simply:

\begin{sphinxVerbatim}[commandchars=\\\{\}]
username@machine\PYGZdl{} ./s2Dcd\PYGZus{}run\PYGZhy{}ex01.py
\end{sphinxVerbatim}

You might need to provide the execution rights to your user
(for example \sphinxcode{\sphinxupquote{chmod +x s2Dcd\_run\sphinxhyphen{}ex01.py}} ).

If you want, you can redirect the output to some file and submit the process
in background:

\begin{sphinxVerbatim}[commandchars=\\\{\}]
username@machine\PYGZdl{} ./s2Dcd\PYGZus{}run\PYGZhy{}ex01.py \PYGZgt{} somefile.out \PYGZam{}
\end{sphinxVerbatim}


\section{Commented example}
\label{\detokenize{examples:commented-example}}
This is a commented version of the file \sphinxcode{\sphinxupquote{s2Dcd\_run\sphinxhyphen{}ex02.py}} that you
can find in the \sphinxcode{\sphinxupquote{..\textbackslash{}examples\textbackslash{}02\_Strebelle\sphinxhyphen{}conditional}}
directory.  It runs a simulation using two training images and some
conditioning data, both in the form of data points and in the form of
a conditioning slice.  The example presents the use of the \sphinxtitleref{DeeSse}
MPS simulation engine, but the principles to run a simulation with a
different MPS simulation engine (i.e. \sphinxtitleref{Impala}) are similar.

\begin{sphinxadmonition}{note}{Note:}
In python, all the text enclosed by \sphinxcode{\sphinxupquote{\textquotesingle{}\textquotesingle{}\textquotesingle{}}}, \sphinxcode{\sphinxupquote{"""}} and the lines
starting with \sphinxcode{\sphinxupquote{\#}} are comments.
\end{sphinxadmonition}

In the first row we define the python interpreter… this is a
standard command that can be included in each python script:

\begin{sphinxVerbatim}[commandchars=\\\{\}]
\PYG{c+ch}{\PYGZsh{}!/usr/bin/env python3}
\end{sphinxVerbatim}

Import some \sphinxstyleemphasis{standard} python modules:

\begin{sphinxVerbatim}[commandchars=\\\{\}]
\PYG{k+kn}{import} \PYG{n+nn}{os}
\PYG{k+kn}{import} \PYG{n+nn}{time}
\PYG{k+kn}{import} \PYG{n+nn}{sys}
\PYG{k+kn}{import} \PYG{n+nn}{numpy} \PYG{k}{as} \PYG{n+nn}{np}
\PYG{k+kn}{import} \PYG{n+nn}{random}
\PYG{k+kn}{import} \PYG{n+nn}{copy}
\end{sphinxVerbatim}

Import the modules which are part of the \sphinxtitleref{s2Dcd} software, that is:

\begin{sphinxVerbatim}[commandchars=\\\{\}]
\PYG{k+kn}{import} \PYG{n+nn}{s2Dcd}\PYG{n+nn}{.}\PYG{n+nn}{s2Dcd} \PYG{k}{as} \PYG{n+nn}{s2Dcd}
\end{sphinxVerbatim}

a module that contains all the interface for the \sphinxtitleref{DeeSse} MPS
simulation engine:

\begin{sphinxVerbatim}[commandchars=\\\{\}]
\PYG{k+kn}{import} \PYG{n+nn}{s2Dcd}\PYG{n+nn}{.}\PYG{n+nn}{deesse} \PYG{k}{as} \PYG{n+nn}{mpds\PYGZus{}interface}
\end{sphinxVerbatim}

\begin{sphinxadmonition}{note}{Note:}
If you want to use a different MPS engine \sphinxtitleref{Impala}, you need to create your own module for this.
\end{sphinxadmonition}

a simple module containing some utilities:

\begin{sphinxVerbatim}[commandchars=\\\{\}]
\PYG{k+kn}{import} \PYG{n+nn}{s2Dcd}\PYG{n+nn}{.}\PYG{n+nn}{utili} \PYG{k}{as} \PYG{n+nn}{utili}
\end{sphinxVerbatim}

a module to read and write output in the GSLIB format:

\begin{sphinxVerbatim}[commandchars=\\\{\}]
\PYG{k+kn}{import} \PYG{n+nn}{s2Dcd}\PYG{n+nn}{.}\PYG{n+nn}{gslibnumpy} \PYG{k}{as} \PYG{n+nn}{gslibnumpy}
\end{sphinxVerbatim}

If you can use a multi\sphinxhyphen{}thread version of \sphinxtitleref{DeeSse}, you can define here
the number of threads:

\begin{sphinxVerbatim}[commandchars=\\\{\}]
\PYG{n}{s2Dcd}\PYG{o}{.}\PYG{n}{nb\PYGZus{}threads} \PYG{o}{=} \PYG{l+m+mi}{4}
\end{sphinxVerbatim}

if you don’t specify a value for this variable, the default value of 1
is used.

Then, the following line is used to print out some information about
the run and record the start time:

\begin{sphinxVerbatim}[commandchars=\\\{\}]
\PYG{n}{time\PYGZus{}start} \PYG{o}{=} \PYG{n}{utili}\PYG{o}{.}\PYG{n}{print\PYGZus{}start}\PYG{p}{(}\PYG{p}{)}
\end{sphinxVerbatim}

A random seed can be set with the following command, in order to allow
to get repeatable results:

\begin{sphinxVerbatim}[commandchars=\\\{\}]
\PYG{n}{seed} \PYG{o}{=} \PYG{n}{numpy}\PYG{o}{.}\PYG{n}{random}\PYG{o}{.}\PYG{n}{RandomState}\PYG{p}{(}\PYG{l+m+mi}{456833}\PYG{p}{)}
\end{sphinxVerbatim}

Then you have to load the default definition for the parameters for
the \sphinxtitleref{DeeSse} simulation. In our case, we set all the parameters in a
template file with in mind the resulting 3D simulation. Therefore, the
grid size and the definition of the search template will be provided
as we would do a 3D simulation.  Here all the parameters are contained
in a standard \sphinxcode{\sphinxupquote{.in}} file for \sphinxtitleref{DeeSse}.  We therefore create a the
template of parameters (\sphinxcode{\sphinxupquote{par\_template}}) from the file
\sphinxcode{\sphinxupquote{template.in}}:

\begin{sphinxVerbatim}[commandchars=\\\{\}]
\PYG{n}{template\PYGZus{}in} \PYG{o}{=} \PYG{l+s+s2}{\PYGZdq{}}\PYG{l+s+s2}{template.in}\PYG{l+s+s2}{\PYGZdq{}}
\PYG{n}{par\PYGZus{}template} \PYG{o}{=} \PYG{n}{mpds\PYGZus{}interface}\PYG{o}{.}\PYG{n}{Param}\PYG{p}{(}\PYG{n}{file\PYGZus{}name}\PYG{o}{=}\PYG{n}{template\PYGZus{}in}\PYG{p}{)}
\end{sphinxVerbatim}

\begin{sphinxadmonition}{note}{Note:}
You could also call \sphinxcode{\sphinxupquote{Param}} without the argument
\sphinxcode{\sphinxupquote{file\_name}}, and you will get the default parameters defined in
the \sphinxcode{\sphinxupquote{mpds\_interface}}.
\end{sphinxadmonition}

Now we define the parameters which are used for the simulation along
the \sphinxtitleref{x}, \sphinxtitleref{y} and \sphinxtitleref{z} axes. We always read the default parameters from
the file defined by the variable \sphinxcode{\sphinxupquote{template\_in}}, but we could
potentially select a different set of parameters for each direction
(and each step of the sequence). Here we keep the parameters defined
in the default \sphinxcode{\sphinxupquote{template.in}} file. However, we customize them
defining a different training image for each direction.  The value
\sphinxcode{\sphinxupquote{None}} must be specified if we don’t have a TI for the plane normal
to that simulation direction:

\begin{sphinxVerbatim}[commandchars=\\\{\}]
\PYG{n}{par\PYGZus{}Xnorm} \PYG{o}{=} \PYG{n}{mpds\PYGZus{}interface}\PYG{o}{.}\PYG{n}{Param}\PYG{p}{(}\PYG{n}{file\PYGZus{}name}\PYG{o}{=}\PYG{n}{template\PYGZus{}in}\PYG{p}{)}
\PYG{n}{par\PYGZus{}Xnorm}\PYG{o}{.}\PYG{n}{tis}\PYG{p}{[}\PYG{l+m+mi}{0}\PYG{p}{]}\PYG{o}{.}\PYG{n}{file\PYGZus{}name} \PYG{o}{=} \PYG{k+kc}{None}
\PYG{n}{par\PYGZus{}Ynorm} \PYG{o}{=} \PYG{n}{mpds\PYGZus{}interface}\PYG{o}{.}\PYG{n}{Param}\PYG{p}{(}\PYG{n}{file\PYGZus{}name}\PYG{o}{=}\PYG{n}{template\PYGZus{}in}\PYG{p}{)}
\PYG{n}{par\PYGZus{}Ynorm}\PYG{o}{.}\PYG{n}{tis}\PYG{p}{[}\PYG{l+m+mi}{0}\PYG{p}{]}\PYG{o}{.}\PYG{n}{file\PYGZus{}name} \PYG{o}{=} \PYG{l+s+s2}{\PYGZdq{}}\PYG{l+s+s2}{ti\PYGZus{}250x1x250.gslib}\PYG{l+s+s2}{\PYGZdq{}}
\PYG{n}{par\PYGZus{}Znorm} \PYG{o}{=} \PYG{n}{mpds\PYGZus{}interface}\PYG{o}{.}\PYG{n}{Param}\PYG{p}{(}\PYG{n}{file\PYGZus{}name}\PYG{o}{=}\PYG{n}{template\PYGZus{}in}\PYG{p}{)}
\PYG{n}{par\PYGZus{}Znorm}\PYG{o}{.}\PYG{n}{tis}\PYG{p}{[}\PYG{l+m+mi}{0}\PYG{p}{]}\PYG{o}{.}\PYG{n}{file\PYGZus{}name} \PYG{o}{=} \PYG{l+s+s2}{\PYGZdq{}}\PYG{l+s+s2}{ti\PYGZus{}250x250x1.gslib}\PYG{l+s+s2}{\PYGZdq{}}
\end{sphinxVerbatim}

Then you have to specify the maximum number of simulation step.
You can usually put this value to a number bigger that the expected
number of slices required to complete your simulation domain. The
simulation will stop when the simulation domain will be filled.

However, in some cases when you would like to run only the fist step
of the simulation, to check is the simulation and the conditioning are
OK, you can set this value to a smaller integer, for example \sphinxcode{\sphinxupquote{6}}.
In this case we select a big value:

\begin{sphinxVerbatim}[commandchars=\\\{\}]
\PYG{n}{step\PYGZus{}max} \PYG{o}{=} \PYG{l+m+mi}{3000}
\end{sphinxVerbatim}

Now we define the simulation grid, which is extracted from the size of
the grid contained in the file \sphinxcode{\sphinxupquote{template.in}} and therefore in the
variable \sphinxcode{\sphinxupquote{par\_template}}:

\begin{sphinxVerbatim}[commandchars=\\\{\}]
\PYG{n}{simODS} \PYG{o}{=} \PYG{n}{par\PYGZus{}template}\PYG{o}{.}\PYG{n}{grid}
\end{sphinxVerbatim}

the grid is printed to the standard output (for double check),
together with other information about the simulation:

\begin{sphinxVerbatim}[commandchars=\\\{\}]
\PYG{n}{s2Dcd}\PYG{o}{.}\PYG{n}{print\PYGZus{}sim\PYGZus{}info}\PYG{p}{(}\PYG{n}{simODS}\PYG{p}{,} \PYG{n}{par\PYGZus{}Xnorm}\PYG{p}{,} \PYG{n}{par\PYGZus{}Ynorm}\PYG{p}{,} \PYG{n}{par\PYGZus{}Znorm}\PYG{p}{)}
\end{sphinxVerbatim}

Then, it is required to define a numpy array which will contain the
results of each step of the simulation. The default values of the
array are initialized using the variable \sphinxcode{\sphinxupquote{s2Dcd.no\_data}}. The
simulation will go on until all the \sphinxtitleref{no\_data} values will be simulated
(or until the maximum simulation step \sphinxcode{\sphinxupquote{step\_max}}):

\begin{sphinxVerbatim}[commandchars=\\\{\}]
\PYG{n}{hard\PYGZus{}data} \PYG{o}{=} \PYG{n}{s2Dcd}\PYG{o}{.}\PYG{n}{no\PYGZus{}data} \PYG{o}{*} \PYG{n}{numpy}\PYG{o}{.}\PYG{n}{ones}\PYG{p}{(}
    \PYG{p}{(}\PYG{n}{simODS}\PYG{o}{.}\PYG{n}{nx}\PYG{p}{,} \PYG{n}{simODS}\PYG{o}{.}\PYG{n}{ny}\PYG{p}{,} \PYG{n}{simODS}\PYG{o}{.}\PYG{n}{nz}\PYG{p}{)}\PYG{p}{,} \PYG{l+s+s1}{\PYGZsq{}}\PYG{l+s+s1}{float}\PYG{l+s+s1}{\PYGZsq{}}\PYG{p}{)}
\end{sphinxVerbatim}

The default value for the variable \sphinxcode{\sphinxupquote{s2Dcd.no\_data}} is \sphinxhyphen{}1.

\begin{sphinxadmonition}{note}{Note:}
You can redefine the value of the variable \sphinxcode{\sphinxupquote{s2Dcd.no\_data}} to
suit different conversions for the no data values.
In this example we consider the default value \sphinxhyphen{}1, therefore a
definition is not required.
If you want to change the no data value you can do something like:

\begin{sphinxVerbatim}[commandchars=\\\{\}]
\PYG{n}{s2Dcd}\PYG{o}{.}\PYG{n}{no\PYGZus{}data} \PYG{o}{=} \PYG{o}{\PYGZhy{}}\PYG{l+m+mi}{999999}
\end{sphinxVerbatim}
\end{sphinxadmonition}

The conditioning data, if available, can be loaded in the GSLIB point
data format using the following command:

\begin{sphinxVerbatim}[commandchars=\\\{\}]
\PYG{n}{s2Dcd}\PYG{o}{.}\PYG{n}{add\PYGZus{}gslib\PYGZus{}pointdata}\PYG{p}{(}\PYG{p}{[}\PYG{l+s+s2}{\PYGZdq{}}\PYG{l+s+s2}{data\PYGZus{}points.gslib}\PYG{l+s+s2}{\PYGZdq{}}\PYG{p}{]}\PYG{p}{,} \PYG{n}{hard\PYGZus{}data}\PYG{p}{,} \PYG{n}{simODS}\PYG{p}{)}
\end{sphinxVerbatim}

\begin{sphinxadmonition}{note}{Note:}
If you have many files containing conditioning data in the
GSLIB point data format, then you can add them at the same time
like this:

\begin{sphinxVerbatim}[commandchars=\\\{\}]
\PYG{n}{s2Dcd}\PYG{o}{.}\PYG{n}{add\PYGZus{}gslib\PYGZus{}pointdata}\PYG{p}{(}\PYG{p}{[}\PYG{l+s+s2}{\PYGZdq{}}\PYG{l+s+s2}{file1.gslib}\PYG{l+s+s2}{\PYGZdq{}}\PYG{p}{,}
    \PYG{l+s+s2}{\PYGZdq{}}\PYG{l+s+s2}{file2.gslib}\PYG{l+s+s2}{\PYGZdq{}}\PYG{p}{,}\PYG{l+s+s2}{\PYGZdq{}}\PYG{l+s+s2}{file3.gslib}\PYG{l+s+s2}{\PYGZdq{}}\PYG{p}{]}\PYG{p}{,} \PYG{n}{hard\PYGZus{}data}\PYG{p}{,} \PYG{n}{simODS}\PYG{p}{)}
\end{sphinxVerbatim}
\end{sphinxadmonition}

With this command the data points contained in the file
\sphinxcode{\sphinxupquote{data\_points.gslib}} will be added to the numpy array that contains
all the results. They will also be extracted at each simulation step
if they are located in the simulated section.  You can also add
conditioning data in a structured grid format, for example if you have
access to an outcrop or some data saved as a GSLIB “image”.  With the
following command, we load the file \sphinxcode{\sphinxupquote{ti\_250x1x250.gslib}} and we
associate its values to the section with \sphinxtitleref{index} 4 along the \sphinxtitleref{y}
coordinate (the exact location depends in the definition of your
grid…):

\begin{sphinxVerbatim}[commandchars=\\\{\}]
\PYG{n}{hd\PYGZus{}section} \PYG{o}{=} \PYG{n}{gslibnumpy}\PYG{o}{.}\PYG{n}{gslib2numpy}\PYG{p}{(}\PYG{l+s+s2}{\PYGZdq{}}\PYG{l+s+s2}{ti\PYGZus{}250x1x250.gslib}\PYG{l+s+s2}{\PYGZdq{}}\PYG{p}{)}
\PYG{n}{hard\PYGZus{}data}\PYG{p}{[}\PYG{p}{:}\PYG{p}{,}\PYG{l+m+mi}{4}\PYG{p}{,}\PYG{p}{:}\PYG{p}{]} \PYG{o}{=} \PYG{n}{hd\PYGZus{}section}\PYG{p}{[}\PYG{l+m+mi}{0}\PYG{p}{:}\PYG{n}{simODS}\PYG{o}{.}\PYG{n}{nx}\PYG{p}{,}\PYG{l+m+mi}{0}\PYG{p}{,}\PYG{l+m+mi}{0}\PYG{p}{:}\PYG{n}{simODS}\PYG{o}{.}\PYG{n}{nz}\PYG{p}{]}
\end{sphinxVerbatim}

Now we can create the simulation sequence. Basically, we alternatively
simulate along the directions defined by the parameters defined by
\sphinxcode{\sphinxupquote{par\_Xnorm}}, \sphinxcode{\sphinxupquote{par\_Ynorm}}, and \sphinxcode{\sphinxupquote{par\_Znorm}} only when a training
image file is present. In this case we didn’t provided a value for the
TI file for the sections perpendicular to the plane \sphinxtitleref{yz} (that is the
parameters contained in \sphinxcode{\sphinxupquote{par\_Xnorm}}). Moreover, for a given
direction we always use the same parameters along the duration of the
simulation sequence. Therefore, to initialize the simulation sequence
we only need:

\begin{sphinxVerbatim}[commandchars=\\\{\}]
\PYG{n}{seq} \PYG{o}{=} \PYG{n}{s2Dcd}\PYG{o}{.}\PYG{n}{create\PYGZus{}seq}\PYG{p}{(}\PYG{n}{simODS}\PYG{p}{,} \PYG{n}{par\PYGZus{}Xnorm}\PYG{p}{,} \PYG{n}{par\PYGZus{}Ynorm}\PYG{p}{,} \PYG{n}{par\PYGZus{}Znorm}\PYG{p}{)}
\end{sphinxVerbatim}

Then comes the true simulation step, which uses the information
collected in the previous ones:

\begin{sphinxVerbatim}[commandchars=\\\{\}]
\PYG{n}{s2Dcd}\PYG{o}{.}\PYG{n}{sim\PYGZus{}run}\PYG{p}{(}\PYG{n}{seq}\PYG{p}{,} \PYG{n}{step\PYGZus{}max}\PYG{p}{,} \PYG{n}{hard\PYGZus{}data}\PYG{p}{,} \PYG{n}{simODS}\PYG{p}{,} \PYG{n}{par\PYGZus{}template}\PYG{p}{,} \PYG{n}{seed}\PYG{p}{)}
\end{sphinxVerbatim}

And finally, compute the running time (approximated!) and print \sphinxtitleref{STOP}:

\begin{sphinxVerbatim}[commandchars=\\\{\}]
\PYG{n}{utili}\PYG{o}{.}\PYG{n}{print\PYGZus{}stop}\PYG{p}{(}\PYG{n}{time\PYGZus{}start}\PYG{p}{)}
\end{sphinxVerbatim}


\section{Examples with \sphinxtitleref{DeeSse}}
\label{\detokenize{examples:examples-with-deesse}}

\subsection{Example 1: Strebelle’s TI along two directions}
\label{\detokenize{examples:example-1-strebelle-s-ti-along-two-directions}}\begin{quote}\begin{description}
\item[{file name}] \leavevmode
\sphinxtitleref{s2Dcd\_run\sphinxhyphen{}ex01.py}

\item[{directory}] \leavevmode
\sphinxtitleref{/examples/deesse/01\_Strebelle}.

\end{description}\end{quote}

In this example, the \sphinxtitleref{s2Dcd} method is applied to the celebrated
Strebelle’s 2D training image to obtain a 3D simulation.
Here we use the same training images along the directions normal to the axis
\sphinxstyleemphasis{z} and \sphinxstyleemphasis{y}.

No conditioning data are considered.

\begin{sphinxadmonition}{note}{Note:}
Another possibility could be, for example, to use the two TI normal to
axis \sphinxstyleemphasis{x} and \sphinxstyleemphasis{y}. In this case the hypothesis about the 3D geometry are
\end{sphinxadmonition}


\subsection{Example 2: Conditioning data}
\label{\detokenize{examples:example-2-conditioning-data}}\begin{quote}\begin{description}
\item[{file name}] \leavevmode
\sphinxtitleref{s2Dcd\_run\sphinxhyphen{}ex02.py}

\item[{directory}] \leavevmode
\sphinxtitleref{/examples/02\_Strebelle\sphinxhyphen{}conditional}.

\end{description}\end{quote}

Same as \sphinxtitleref{Example 1}, but with conditioning points.


\subsection{Example 3: Multiple realizations}
\label{\detokenize{examples:example-3-multiple-realizations}}\begin{quote}\begin{description}
\item[{file name}] \leavevmode
\sphinxtitleref{s2Dcd\_run\sphinxhyphen{}ex03.py}

\item[{directory}] \leavevmode
\sphinxtitleref{/examples/03\_Strebelle\sphinxhyphen{}many\sphinxhyphen{}realizations}.

\end{description}\end{quote}

In this example, the same simulation framework of \sphinxtitleref{Example 1} is applied
many times in order to obtain different simulations with different
random seeds.


\subsection{Example 4: Changing the simulation parameters}
\label{\detokenize{examples:example-4-changing-the-simulation-parameters}}\begin{quote}\begin{description}
\item[{file name}] \leavevmode
\sphinxtitleref{s2Dcd\_run\sphinxhyphen{}ex01.py}

\item[{directory}] \leavevmode
\sphinxtitleref{/examples/04\_Strebelle\sphinxhyphen{}many\sphinxhyphen{}param}.

\end{description}\end{quote}

The modules related to the \sphinxtitleref{s2Dcd} can be used also for other
purposes, like for example to play with the parameters to be used
for a MPS simulation in an automated way.
You can for example use the modules to run different simulations
changing automatically the size of the data template.
Here we demonstrate this using the Strebelle’s training image.

No conditioning data are considered.


\chapter{Notes}
\label{\detokenize{notes:notes}}\label{\detokenize{notes::doc}}

\section{Auxiliary variables}
\label{\detokenize{notes:auxiliary-variables}}
The information provided in this section are useful to set up the
template file and the \sphinxstyleemphasis{Python} script useful to run the \sphinxcode{\sphinxupquote{s2Dcd}}
using auxiliary variables.


\subsection{Implementation in DeeSse}
\label{\detokenize{notes:implementation-in-deesse}}
\begin{sphinxadmonition}{note}{Note:}
This option might work properly, but it is in a early stage
development.
\end{sphinxadmonition}

The use of auxiliary variables with the \sphinxtitleref{DeeSse}
MPS engine could be implemented in a number of ways. Here the
following strategy is selected:
\begin{enumerate}
\sphinxsetlistlabels{\arabic}{enumi}{enumii}{}{)}%
\item {} 
The 2D slices (primary and auxiliary variable) are provided in the
\sphinxstyleemphasis{same} \sphinxtitleref{GSLIB} file using the variable name provided in the
\sphinxcode{\sphinxupquote{template.in}}
file. In other words, only one TI is defined in the section \sphinxcode{\sphinxupquote{TRAINING
IMAGE}}.

\item {} 
The intermediate output files are provided as a unique file
too. Therefore, in the section \sphinxcode{\sphinxupquote{OUTPUT SETTINGS FOR SIMULATION}}
in the \sphinxcode{\sphinxupquote{template.in}} file the keyword
\sphinxcode{\sphinxupquote{OUTPUT\_SIM\_ALL\_IN\_ONE\_FILE}} should be used.

\end{enumerate}


\subsection{Implementation in Impala}
\label{\detokenize{notes:implementation-in-impala}}
\begin{sphinxadmonition}{warning}{Warning:}
This MPS engine is not supported any more
since 2017. This section is left just as a reference for
further developments, potentially oriented towards other
MPS simulation engines.
\end{sphinxadmonition}

There are two main modes that can be implemented to use auxiliary variables:
\begin{description}
\item[{Full 3D mode}] \leavevmode
In this mode a full 3D auxliary variable map must be
provided. Then the \sphinxtitleref{s2Dcd} automatically slices it along the
current simulated section to extract a 2D auxiliary variable
map. This set up is useful when it is relatively easy to find a
full 3D map for the auxiliary variable.

\item[{2D mode}] \leavevmode
This should be the default mode, when you use the 2D maps of
auxiliary variables attached to the 2D TIs provided for the
simulation.

\end{description}

The \sphinxtitleref{Full 3D mode} can be enabled by setting the
\sphinxcode{\sphinxupquote{geostats.s2Dcd.AUX\_VAR\_FULL3D}} parameter to \sphinxcode{\sphinxupquote{True}}. Otherwise,
the default value for the \sphinxtitleref{2D mode} is used.

For the \sphinxcode{\sphinxupquote{Impala}} MPS simulation engine the implementation of usage
of the auxiliary variable is quite univoque and follows a quite
standard workflow.  Simply, the user have to provide (in addition to
the 2D TIs of the main variable):
\begin{itemize}
\item {} 
a 2D auxiliary variable map (\sphinxtitleref{VTK} files) with the same size of
the provided 2D TIs, one for each TI.

\item {} 
a 3D auxiliary variable map with the same size of the 3D simulation
grid.

\end{itemize}

\begin{sphinxadmonition}{note}{Note:}\begin{enumerate}
\sphinxsetlistlabels{\arabic}{enumi}{enumii}{}{)}%
\item {} 
The format of the variable in the \sphinxstyleemphasis{VTK} files contaning the
auxiliary variables should be \sphinxcode{\sphinxupquote{float}}.

\item {} 
During the simulation process many \sphinxstyleemphasis{VTK} files containing
sections of the 3D auxiliary variable map will be created. This
is somehow redundant and in the future should be removed. For
the moment you have to manually remove manually these files.

\item {} 
Very often the information contained in the 3D auxiliary
variable sections is not very different from the 2D auxiliary
variable corresponding to the 2D TI. Some computation time
could be saved in these cases, but for the moment a complete 3D
auxliary variable map is required to provide more flexibility.

\end{enumerate}
\end{sphinxadmonition}


\section{Licence Issues}
\label{\detokenize{notes:licence-issues}}
The recent versions of the DS codes are running with a licence
manager. Therefore, it is possible that running the code will give an
error code. At the moment the quick and dirty solution is to wait for
some time and retry to run again the simulation of the same
section. You can tune this in the module \sphinxcode{\sphinxupquote{deesse.py}}, changing the
values of the variables \sphinxcode{\sphinxupquote{NB\_LIC\_WAIT}} (number of attempts to contact
the lincese server) and \sphinxcode{\sphinxupquote{LIC\_WAIT\_TIME}} (pause from one attempt to the
other).

\begin{sphinxadmonition}{warning}{Warning:}
Note that the computing time can be heavily affected by
this problem!
\end{sphinxadmonition}


\chapter{Publications}
\label{\detokenize{publications:publications}}\label{\detokenize{publications::doc}}
Some pubblications where the \sphinxtitleref{s2Dcd} was used:
\begin{itemize}
\item {} 
A.Pickel, J.D.Frechette, \sphinxstylestrong{A.Comunian} and G.S.Weissmann (2015)
“Building a Better Training Image with Digital Outcrop Models” \sphinxhyphen{}
Journal of Hydrology 531(part 1) \sphinxhref{http://dx.doi.org/10.1016/j.jhydrol.2015.08.049}{DOI:
10.1016/j.jhydrol.2015.08.049}

\item {} 
P.Bayer, \sphinxstylestrong{A.Comunian}, D.Höyng, \& G.Mariethoz
(2015) “High resolution multi\sphinxhyphen{}facies realizations of sedimentary reservoir
and aquifer analogs” \sphinxhyphen{} Scientific Data, 2 \sphinxhref{http://dx.doi.org/10.1038/sdata.2015.33}{DOI: 10.1038/sdata.2015.33} \sphinxhyphen{} 

\item {} 
T.C.Kessler, \sphinxstylestrong{A.Comunian}, F.Oriani, P.Renard, B.Nilsson,
K.E.Klint and P.L.Bjerg (2013) “Modeling Fine\sphinxhyphen{}Scale Geological
Heterogeneity \sphinxhyphen{} Examples of Sand Lenses in Tills.”  Groundwater
51(5) DOI: \sphinxhref{http://dx.doi.org/10.1111/j.1745-6584.2012.01015.x}{10.1111/j.1745\sphinxhyphen{}6584.2012.01015.x}

\item {} 
\sphinxstylestrong{A.Comunian}, P.Renard and J.Straubhaar (2012)
“3D multiple\sphinxhyphen{}point statistics simulation using 2D training images.”
Computers \& Geosciences   40
DOI: \sphinxhref{http://dx.doi.org/10.1016/j.cageo.2011.07.009}{10.1016/j.cageo.2011.07.009}
\sphinxhyphen{} 

\item {} 
N.Gueting, J.Caers, \sphinxstylestrong{A.Comunian}, J.Vanderborght, A.Englert,
“Reconstruction of Three\sphinxhyphen{}Dimensional Aquifer Heterogeneity from
Two\sphinxhyphen{}Dimensional Geophysical Data” \sphinxhyphen{} Mathematical Geosciences
\sphinxhyphen{} DOI: \sphinxhref{http://dx.doi.org/10.1007/s11004-017-9694-x}{10.1007/s11004\sphinxhyphen{}017\sphinxhyphen{}9694\sphinxhyphen{}x} \sphinxhyphen{} 

\item {} 
Q.Chen, G.Mariethoz, G.Liu, \sphinxstylestrong{A.Comunian}, and X.Ma
“Locality\sphinxhyphen{}based 3\sphinxhyphen{}D multiple\sphinxhyphen{}point statistics reconstruction using 2\sphinxhyphen{}D geological cross\sphinxhyphen{}sections” \sphinxhyphen{}
Hydrology and Earth System Sciences \sphinxhyphen{} DOI: \sphinxhref{https://doi.org/10.5194/hess-2018-256}{10.5194/hess\sphinxhyphen{}2018\sphinxhyphen{}256} \sphinxhyphen{} 

\end{itemize}


\chapter{Appendices}
\label{\detokenize{appendices:appendices}}\label{\detokenize{appendices:id1}}\label{\detokenize{appendices::doc}}
Here some additional documentation about the \sphinxtitleref{Python} modules used by
the \sphinxtitleref{s2Dcd} tool.


\section{The \sphinxstyleliteralintitle{\sphinxupquote{s2Dcd}} module}
\label{\detokenize{appendices:module-s2Dcd.s2Dcd}}\label{\detokenize{appendices:the-s2dcd-module}}\index{s2Dcd.s2Dcd (module)@\spxentry{s2Dcd.s2Dcd}\spxextra{module}}\begin{quote}\begin{description}
\item[{license}] \leavevmode
This file is part of s2Dcd.

s2Dcd is free software: you can redistribute it and/or modify
it under the terms of the GNU General Public License as published by
the Free Software Foundation, either version 3 of the License, or
(at your option) any later version.

s2Dcd is distributed in the hope that it will be useful,
but WITHOUT ANY WARRANTY; without even the implied warranty of
MERCHANTABILITY or FITNESS FOR A PARTICULAR PURPOSE.  See the
GNU General Public License for more details.

You should have received a copy of the GNU General Public License
along with s2Dcd.  If not, see \textless{}\sphinxurl{https://www.gnu.org/licenses/}\textgreater{}.

\item[{Purpose}] \leavevmode
A module to apply the \sphinxtitleref{s2Dcd} multiple\sphinxhyphen{}point simulation approach.
At the moment it is implemented with the \sphinxtitleref{DeeSse} MPS simulation engine,
but could be adapted easily to other MPS engines.

\item[{File name}] \leavevmode
\sphinxcode{\sphinxupquote{s2Dcd.py}}

\item[{Version}] \leavevmode\begin{description}
\item[{0.9.9 , 2018\sphinxhyphen{}01\sphinxhyphen{}23 :}] \leavevmode\begin{itemize}
\item {} 
Solved one bug that prevented simulation perpendicular
to axis \sphinxstyleemphasis{x}.

\end{itemize}

\item[{0.9.8 , 2017\sphinxhyphen{}12\sphinxhyphen{}15 :}] \leavevmode\begin{itemize}
\item {} 
Solved one but related to casting some index.

\end{itemize}

\item[{0.9.6 , 2014\sphinxhyphen{}08\sphinxhyphen{}12 :}] \leavevmode\begin{itemize}
\item {} 
Try to include the auxiliary variables treatment.

\item {} 
Changed name of the module related to the DS.

\end{itemize}

\item[{0.9.5 , 2013\sphinxhyphen{}11\sphinxhyphen{}15 :}] \leavevmode\begin{itemize}
\item {} 
Converted to Python3 with \sphinxtitleref{2to3}.

\item {} 
Corrected a bug in the function \sphinxtitleref{matrioska\_interval}.

\item {} 
Adapted to the new version of gslibnumpy.

\end{itemize}

\item[{0.9.4 , 2012\sphinxhyphen{}09\sphinxhyphen{}05 :}] \leavevmode\begin{itemize}
\item {} 
Last version before the movement to Python3

\end{itemize}

\item[{0.9.3 , 2012\sphinxhyphen{}09\sphinxhyphen{}04 :}] \leavevmode\begin{itemize}
\item {} 
Tested on a simple case study with one thread and no
auxiliary variables.

\end{itemize}

\item[{0.9.2, 2012\sphinxhyphen{}05\sphinxhyphen{}02 :}] \leavevmode\begin{itemize}
\item {} 
Defined some variables to describe the range of integer values
that the seed should take.

\item {} 
Corrected a bug related to bad parenthesis (thank you Andrea!)

\end{itemize}

\item[{0.9.1,  2012\sphinxhyphen{}05\sphinxhyphen{}02 :}] \leavevmode\begin{itemize}
\item {} 
Added an option to better define when a problem is for categorical
variables or for continuous ones.

\item {} 
Solved a bug (forgotten to print the \sphinxtitleref{in} file in the case
of multiple threads).

\end{itemize}

\item[{0.8,  2012\sphinxhyphen{}04\sphinxhyphen{}02 :}] \leavevmode\begin{itemize}
\item {} 
Modified some structures in order to include them in a
module containing some tools to deal with structured
grids… in order to prepare the interface to the MPDS code.

\end{itemize}

\item[{0.7,  2012\sphinxhyphen{}02\sphinxhyphen{}24 :}] \leavevmode\begin{itemize}
\item {} 
Using the variable \sphinxcode{\sphinxupquote{no\_data}} the default value for
the not yet simulated nodes can be re\sphinxhyphen{}defined with more
flexibility.

\end{itemize}

\item[{0.6,  2012\sphinxhyphen{}02\sphinxhyphen{}23 :}] \leavevmode\begin{itemize}
\item {} 
Added a functionality allowing to define a random seed.

\item {} 
Added a function to create a simple \sphinxstyleemphasis{standard} simulation
sequence.

\end{itemize}

\item[{0.5,  2012\sphinxhyphen{}02\sphinxhyphen{}16 :}] \leavevmode\begin{itemize}
\item {} 
added the class {\hyperref[\detokenize{appendices:s2Dcd.s2Dcd.SeqStep}]{\sphinxcrossref{\sphinxcode{\sphinxupquote{SeqStep}}}}} to  simplify notation.

\end{itemize}

\item[{0.4,  2012\sphinxhyphen{}02\sphinxhyphen{}16 :}] \leavevmode\begin{itemize}
\item {} 
cleaned up some procedures and documentation.

\end{itemize}

\item[{0.3, :}] \leavevmode\begin{itemize}
\item {} 
implemented the treatment of auxiliary variables.

\end{itemize}

\item[{0.2, :}] \leavevmode\begin{itemize}
\item {} 
some improvements…

\end{itemize}

\item[{0.1, :}] \leavevmode\begin{itemize}
\item {} 
first version.

\end{itemize}

\end{description}

\item[{Authors}] \leavevmode
Alessandro Comunian.

\end{description}\end{quote}


\subsection{Usage}
\label{\detokenize{appendices:usage}}
See the examples in the corresponding directory and the documentation
of the main functions.


\subsection{Limitations}
\label{\detokenize{appendices:limitations}}\begin{description}
\item[{auxiliary variables:}] \leavevmode
For the moment they are implemented only for the \sphinxcode{\sphinxupquote{implala}} MPS
engine.

\item[{number of variables:}] \leavevmode
\sphinxtitleref{MPDS} can be used with only one variable.

\end{description}


\subsection{TODO}
\label{\detokenize{appendices:todo}}\begin{itemize}
\item {} 
Include the usage of auxiliary variables when using all the MPS
engines.

\item {} 
Implement all the features that can be described in the parameter
files.

\item {} 
Deal with the connection problem that can raise when the licence
cannot be verified or the connection is slow.

\item {} 
Improve the output and input format, adding for example the SGeMS
binary input/output format (look for the format in mGstat, for
example)

\item {} 
For some simulation engines (like the one based on the snesim/impala engines), 
that can store information coming from threes/lists in and external file, it can be 
useful to allow the re\sphinxhyphen{}use of these.

\item {} 
Add some functionality to delete all the output files created,
something like \sphinxstyleemphasis{DEBUG} or \sphinxstyleemphasis{VERBOSE} mode.

\end{itemize}
\index{GetWhereToAddData() (in module s2Dcd.s2Dcd)@\spxentry{GetWhereToAddData()}\spxextra{in module s2Dcd.s2Dcd}}

\begin{fulllineitems}
\phantomsection\label{\detokenize{appendices:s2Dcd.s2Dcd.GetWhereToAddData}}\pysiglinewithargsret{\sphinxcode{\sphinxupquote{s2Dcd.s2Dcd.}}\sphinxbfcode{\sphinxupquote{GetWhereToAddData}}}{\emph{vtkReaderOutput}}{}
Get where the data file should be added into the hard data
archive.
\begin{description}
\item[{Parameters:}] \leavevmode\begin{description}
\item[{vtkReaderOutput:}] \leavevmode
The result of a “GetOutput()” from a
“vtkStructuredPointsReader” object.

\end{description}

\item[{Returns:}] \leavevmode\begin{itemize}
\item {} 
The axis normal to the plane where the simulation was
performed in the format character, that is \sphinxstyleemphasis{x}, \sphinxstyleemphasis{y} or \sphinxstyleemphasis{z}.

\item {} 
the “coordinate” (in the reference system of the simulation,
that is related to the “pixel” which have to be simulated in
the matrix of hard data) of the slice where the new
simulated data should be added.

\end{itemize}

\end{description}

\begin{sphinxadmonition}{warning}{Warning:}
This function is obsolete.
\end{sphinxadmonition}

\end{fulllineitems}

\index{SeqStep (class in s2Dcd.s2Dcd)@\spxentry{SeqStep}\spxextra{class in s2Dcd.s2Dcd}}

\begin{fulllineitems}
\phantomsection\label{\detokenize{appendices:s2Dcd.s2Dcd.SeqStep}}\pysiglinewithargsret{\sphinxbfcode{\sphinxupquote{class }}\sphinxcode{\sphinxupquote{s2Dcd.s2Dcd.}}\sphinxbfcode{\sphinxupquote{SeqStep}}}{\emph{direct}, \emph{level}, \emph{param}, \emph{pseudo3D=0}}{}
A class that contains all the information required to create a
parameters input file for the used MPS engines, that is a \sphinxstyleemphasis{step} of
the simulation sequence of the approach \sphinxtitleref{s2Dcd}.
\index{create\_list() (s2Dcd.s2Dcd.SeqStep method)@\spxentry{create\_list()}\spxextra{s2Dcd.s2Dcd.SeqStep method}}

\begin{fulllineitems}
\phantomsection\label{\detokenize{appendices:s2Dcd.s2Dcd.SeqStep.create_list}}\pysiglinewithargsret{\sphinxbfcode{\sphinxupquote{create\_list}}}{\emph{simODS}, \emph{par\_template}}{}
Run Impala to create the lists required along a given
direction.
\begin{description}
\item[{Parameters:}] \leavevmode\begin{description}
\item[{simODS: object of type \sphinxcode{\sphinxupquote{Grid}}}] \leavevmode
Contains all the info concerning the simulation
domain.

\item[{par\_template: class \sphinxcode{\sphinxupquote{impala\_interface.Param}}.}] \leavevmode
The parameters for creating the \sphinxstyleemphasis{.in} file for the
simulations.

\end{description}

\item[{Returns:}] \leavevmode\begin{itemize}
\item {} 
Create for each level of multigrid a list in binary format.

\end{itemize}

\end{description}

\end{fulllineitems}

\index{simul() (s2Dcd.s2Dcd.SeqStep method)@\spxentry{simul()}\spxextra{s2Dcd.s2Dcd.SeqStep method}}

\begin{fulllineitems}
\phantomsection\label{\detokenize{appendices:s2Dcd.s2Dcd.SeqStep.simul}}\pysiglinewithargsret{\sphinxbfcode{\sphinxupquote{simul}}}{\emph{step}, \emph{step\_max}, \emph{hard\_data}, \emph{simODS}, \emph{in\_par}, \emph{seed}, \emph{rcp\_lists}}{}
Run a MPS simulation for a simulation step of a sequence.
\begin{description}
\item[{Parameters:}] \leavevmode\begin{description}
\item[{step: int}] \leavevmode
The current simulation step.

\item[{step\_max: int}] \leavevmode
The maximum simulation step. Note that the simulation
domain can be often filled before this step. In this
case, the simulation procedure is stopped.  Defining
this parameter smaller than the total number of
simulations expected to fill the domain can be useful
for doing some preliminar test.

\item[{hard\_data: numpy array}] \leavevmode
A numpy array containig all the simulation. The value
of the global variable \sphinxcode{\sphinxupquote{no\_data}} is used in
locations not yet simulated.

\item[{simODS: object of type \sphinxcode{\sphinxupquote{Grid}}}] \leavevmode
Contains all the info concerning the simulation
domain.

\item[{in\_par: string or member of the class \sphinxcode{\sphinxupquote{Param}}.}] \leavevmode
The parameters for creating the \sphinxstyleemphasis{.in} file for the
simulations.  In case it is a string: The name of the
file containing the template of the input parameters
file.  In case it is a \sphinxtitleref{Param} object:
All the parameters readed from a template \sphinxtitleref{.in} file
required for a MPDS simulation.  seed: instance of the
class \sphinxcode{\sphinxupquote{RandomState}} To keep track of the
random seed and create simulations that can be
reproduced using the same seed.

\item[{rcp\_lists: flag}] \leavevmode
Useful if one wants to re\sphinxhyphen{}compute the list at each 
simulation step. This parameter has sense only for the
\sphinxtitleref{Impala} family of MPS simulation engines.

\end{description}

\item[{Returns:}] \leavevmode\begin{itemize}
\item {} 
Create some files required for the simulation.

\item {} 
Update the content of the numpy array \sphinxtitleref{hard\_data}
including new simulated nodes.

\end{itemize}

If the selected section is full of hard data, simply
returns 1 without creating files and without running the
MPS simulation.  If there is an error, returns \sphinxhyphen{}1.

\end{description}

\end{fulllineitems}


\end{fulllineitems}

\index{adaptAuxVarFile() (in module s2Dcd.s2Dcd)@\spxentry{adaptAuxVarFile()}\spxextra{in module s2Dcd.s2Dcd}}

\begin{fulllineitems}
\phantomsection\label{\detokenize{appendices:s2Dcd.s2Dcd.adaptAuxVarFile}}\pysiglinewithargsret{\sphinxcode{\sphinxupquote{s2Dcd.s2Dcd.}}\sphinxbfcode{\sphinxupquote{adaptAuxVarFile}}}{\emph{seq}, \emph{simODS}}{}
Change the dimension in the VTK file containing the auxiliary
variable map in order to adapt it for the current 2D simulation.
\begin{description}
\item[{Parameters:}] \leavevmode\begin{description}
\item[{seq: object of type {\hyperref[\detokenize{appendices:s2Dcd.s2Dcd.SeqStep}]{\sphinxcrossref{\sphinxcode{\sphinxupquote{SeqStep}}}}}}] \leavevmode
Contains the information related to the current simulation
step.

\item[{simODS: object of type \sphinxcode{\sphinxupquote{Grid}}}] \leavevmode
Contains the dimension required to define a simulation domain.

\end{description}

\item[{Returns:}] \leavevmode
A file containing a auxiliary variable whith dimensions
suitable for the current simulation step.

\end{description}

\end{fulllineitems}

\index{addVtk2HdArchive() (in module s2Dcd.s2Dcd)@\spxentry{addVtk2HdArchive()}\spxextra{in module s2Dcd.s2Dcd}}

\begin{fulllineitems}
\phantomsection\label{\detokenize{appendices:s2Dcd.s2Dcd.addVtk2HdArchive}}\pysiglinewithargsret{\sphinxcode{\sphinxupquote{s2Dcd.s2Dcd.}}\sphinxbfcode{\sphinxupquote{addVtk2HdArchive}}}{\emph{hard\_data}, \emph{facies}}{}
A function to add all the VTK files contained into the current
directory into an hard data archive file (that is a 3D matrix
which once filled in will be the “final” simulation).
\begin{description}
\item[{Parameters:}] \leavevmode\begin{description}
\item[{hard\_data: string}] \leavevmode
Name of the file which contains all the conditioning data.

\item[{facies: string}] \leavevmode
A string containing the facies which are considered.

\end{description}

\end{description}

\begin{sphinxadmonition}{warning}{Warning:}
This function is obsolete. It can be time consuming if a lot
of VTK files are present in the working directory.
\end{sphinxadmonition}

\end{fulllineitems}

\index{add\_gslib\_pointdata() (in module s2Dcd.s2Dcd)@\spxentry{add\_gslib\_pointdata()}\spxextra{in module s2Dcd.s2Dcd}}

\begin{fulllineitems}
\phantomsection\label{\detokenize{appendices:s2Dcd.s2Dcd.add_gslib_pointdata}}\pysiglinewithargsret{\sphinxcode{\sphinxupquote{s2Dcd.s2Dcd.}}\sphinxbfcode{\sphinxupquote{add\_gslib\_pointdata}}}{\emph{data\_files}, \emph{hard\_data}, \emph{simODS}}{}
Add the hard data contained in a number of GSLIB point data files.
\begin{description}
\item[{Parameters:}] \leavevmode\begin{description}
\item[{files: list of strings}] \leavevmode
A number of files containing the data. Note that this must
be in a list format even if only one file is
considered. For example, you should always use a syntax
like \sphinxtitleref{{[}“file1.gslib”{]}} even if you provide only one file.

\item[{hard\_data: 3D numpy array}] \leavevmode
Where all the simulated points are stored.

\item[{simODS: object of type \sphinxcode{\sphinxupquote{Grid}}}] \leavevmode
Contains all the info concerning the simulation domain.

\end{description}

\item[{Returns:}] \leavevmode
Update the content of the array \sphinxtitleref{hard\_data}.

\end{description}

\end{fulllineitems}

\index{add\_hd() (in module s2Dcd.s2Dcd)@\spxentry{add\_hd()}\spxextra{in module s2Dcd.s2Dcd}}

\begin{fulllineitems}
\phantomsection\label{\detokenize{appendices:s2Dcd.s2Dcd.add_hd}}\pysiglinewithargsret{\sphinxcode{\sphinxupquote{s2Dcd.s2Dcd.}}\sphinxbfcode{\sphinxupquote{add\_hd}}}{\emph{seq\_step}, \emph{new\_hd}, \emph{hard\_data}, \emph{simODS}}{}
Add a numpy array containing the data simulated at a given
sequence step to the array that stores the simulated nodes.
\begin{description}
\item[{Parameters:}] \leavevmode\begin{description}
\item[{seq\_step: instance of the class {\hyperref[\detokenize{appendices:s2Dcd.s2Dcd.SeqStep}]{\sphinxcrossref{\sphinxcode{\sphinxupquote{SeqStep}}}}}}] \leavevmode
Information about the current simulation step.

\item[{new\_hd: numpy array (2D)}] \leavevmode
The new simulated section.

\item[{hard\_data: numpy array (3D)}] \leavevmode
The contained of all the simulated data.

\item[{simODS:  object of type \sphinxcode{\sphinxupquote{Grid}}}] \leavevmode
Information about the simulation grid.

\end{description}

\item[{Returns:}] \leavevmode
Fit the content provided with \sphinxtitleref{new\_hd} into the right position
into the array \sphinxtitleref{hard\_data}.

\end{description}

\begin{sphinxadmonition}{note}{Note:}\begin{itemize}
\item {} 
A the moment the fact that a gslib file can contain multiple 
variables is handled with a quick and dirty trick.

\end{itemize}
\end{sphinxadmonition}

\end{fulllineitems}

\index{all\_segms0() (in module s2Dcd.s2Dcd)@\spxentry{all\_segms0()}\spxextra{in module s2Dcd.s2Dcd}}

\begin{fulllineitems}
\phantomsection\label{\detokenize{appendices:s2Dcd.s2Dcd.all_segms0}}\pysiglinewithargsret{\sphinxcode{\sphinxupquote{s2Dcd.s2Dcd.}}\sphinxbfcode{\sphinxupquote{all\_segms0}}}{\emph{segms}}{}
Check if a list of segments contains at least one segment with
length \textgreater{} 0.
\begin{description}
\item[{Parameters:}] \leavevmode\begin{description}
\item[{segms:}] \leavevmode
A list of tuples representing 1D segments.

\end{description}

\item[{Returns:}] \leavevmode
True if all the segments contained into the list have length = 0,
False if at least one segment has a length \textgreater{} 0.

\end{description}

Example

\begin{sphinxVerbatim}[commandchars=\\\{\}]
\PYG{g+gp}{\PYGZgt{}\PYGZgt{}\PYGZgt{} }\PYG{n}{all\PYGZus{}segms0}\PYG{p}{(}\PYG{p}{[}\PYG{p}{(}\PYG{l+m+mi}{1}\PYG{p}{,}\PYG{l+m+mi}{4}\PYG{p}{)}\PYG{p}{,} \PYG{p}{(}\PYG{l+m+mi}{2}\PYG{p}{,}\PYG{l+m+mi}{7}\PYG{p}{)}\PYG{p}{,} \PYG{p}{(}\PYG{l+m+mi}{20}\PYG{p}{,}\PYG{l+m+mi}{20}\PYG{p}{)}\PYG{p}{]}\PYG{p}{)}
\PYG{g+go}{False}
\PYG{g+gp}{\PYGZgt{}\PYGZgt{}\PYGZgt{} }\PYG{n}{all\PYGZus{}segms0}\PYG{p}{(}\PYG{p}{[}\PYG{p}{(}\PYG{l+m+mi}{1}\PYG{p}{,}\PYG{l+m+mi}{1}\PYG{p}{)}\PYG{p}{,} \PYG{p}{(}\PYG{l+m+mi}{7}\PYG{p}{,}\PYG{l+m+mi}{7}\PYG{p}{)}\PYG{p}{,} \PYG{p}{(}\PYG{l+m+mi}{20}\PYG{p}{,}\PYG{l+m+mi}{20}\PYG{p}{)}\PYG{p}{]}\PYG{p}{)}
\PYG{g+go}{True}
\end{sphinxVerbatim}

\end{fulllineitems}

\index{check\_ti\_file() (in module s2Dcd.s2Dcd)@\spxentry{check\_ti\_file()}\spxextra{in module s2Dcd.s2Dcd}}

\begin{fulllineitems}
\phantomsection\label{\detokenize{appendices:s2Dcd.s2Dcd.check_ti_file}}\pysiglinewithargsret{\sphinxcode{\sphinxupquote{s2Dcd.s2Dcd.}}\sphinxbfcode{\sphinxupquote{check\_ti\_file}}}{\emph{par}}{}
Check if a parameter file for \sphinxtitleref{Impala} or \sphinxtitleref{MPDS} has a name for
the TI file or not.
\begin{description}
\item[{Parameters:}] \leavevmode
par: object of type \sphinxtitleref{Param} or \sphinxtitleref{Param}.

\item[{Returns:}] \leavevmode
True if a TI file name is defines, false if None.

\end{description}

\end{fulllineitems}

\index{create\_in\_file4Impala() (in module s2Dcd.s2Dcd)@\spxentry{create\_in\_file4Impala()}\spxextra{in module s2Dcd.s2Dcd}}

\begin{fulllineitems}
\phantomsection\label{\detokenize{appendices:s2Dcd.s2Dcd.create_in_file4Impala}}\pysiglinewithargsret{\sphinxcode{\sphinxupquote{s2Dcd.s2Dcd.}}\sphinxbfcode{\sphinxupquote{create\_in\_file4Impala}}}{\emph{in\_par}, \emph{seq\_si}, \emph{file\_cond}, \emph{file\_name\_sim}, \emph{seed=None}}{}
Create a parameters input file for \sphinxstyleemphasis{Impala}. A big part of the
information is read directly from the template input file for
\sphinxstyleemphasis{Impala}.
\begin{description}
\item[{Parameters:}] \leavevmode\begin{description}
\item[{in\_par: object of type \sphinxtitleref{impala\_interface.Param}.}] \leavevmode
All the information contained in the template file.

\item[{seq\_si: object of the class {\hyperref[\detokenize{appendices:s2Dcd.s2Dcd.SeqStep}]{\sphinxcrossref{\sphinxcode{\sphinxupquote{SeqStep}}}}}.}] \leavevmode
All the informations required to create a simulation for the
current simulation step. See the class {\hyperref[\detokenize{appendices:s2Dcd.s2Dcd.SeqStep}]{\sphinxcrossref{\sphinxcode{\sphinxupquote{SeqStep}}}}}
for details.

\item[{file\_cond: string or None}] \leavevmode
Name of the conditioning file, if None the simulation is
considered as non conditional.

\item[{file\_name\_sim: string}] \leavevmode
Name of the input paramter file for \sphinxstyleemphasis{Impala}.

\item[{seed: instance of the class \sphinxcode{\sphinxupquote{RandomState}}, optional}] \leavevmode
To keep track of the random seed and create simulations
that can be reproduced using the same seed.  A definition
of the seed is not required when the MPS core is called
only for the generation of the lists. Therefore, in this
case the value of seed can be None, and the value of the
module variable \sphinxtitleref{seed\_default} is used.

\end{description}

\item[{Returns:}] \leavevmode
A \sphinxtitleref{*.in} file containing the parameters for running \sphinxtitleref{Impala}.

\end{description}

\begin{sphinxadmonition}{note}{Note:}
In order to increase the variability of the simulations, a new
random seed is generated for each simulation file that is
created if a \sphinxcode{\sphinxupquote{RandomState}} instance is provided with
the paramter \sphinxtitleref{seed}. Otherwise, the value of the global
variable \sphinxtitleref{seed\_default} is used.
\end{sphinxadmonition}

\end{fulllineitems}

\index{create\_in\_file4MPDS() (in module s2Dcd.s2Dcd)@\spxentry{create\_in\_file4MPDS()}\spxextra{in module s2Dcd.s2Dcd}}

\begin{fulllineitems}
\phantomsection\label{\detokenize{appendices:s2Dcd.s2Dcd.create_in_file4MPDS}}\pysiglinewithargsret{\sphinxcode{\sphinxupquote{s2Dcd.s2Dcd.}}\sphinxbfcode{\sphinxupquote{create\_in\_file4MPDS}}}{\emph{in\_par}, \emph{seq\_si}, \emph{file\_cond}, \emph{file\_name\_sim}, \emph{seed=None}}{}
Create a parameters input file for \sphinxtitleref{MPDS} adapted to the current
simulation step and the current conditioning data file.
\begin{description}
\item[{Parameters:}] \leavevmode\begin{description}
\item[{in\_par: object of type \sphinxtitleref{ds\_interface.Param}.}] \leavevmode
All the info about the in template file.

\item[{seq\_si: object of the class {\hyperref[\detokenize{appendices:s2Dcd.s2Dcd.SeqStep}]{\sphinxcrossref{\sphinxcode{\sphinxupquote{SeqStep}}}}}.}] \leavevmode
(sequence stepA info) All the informations required to
create a simulation for the current simulation step. See
the class {\hyperref[\detokenize{appendices:s2Dcd.s2Dcd.SeqStep}]{\sphinxcrossref{\sphinxcode{\sphinxupquote{SeqStep}}}}} for details.

\item[{file\_cond: string or None}] \leavevmode
Name of the conditioning file, if None the simulation is
considered as non conditional.

\item[{file\_name\_sim: string}] \leavevmode
Name of the input paramter file for \sphinxtitleref{MPDS}.

\item[{seed: instance of the class \sphinxcode{\sphinxupquote{RandomState}}, optional}] \leavevmode
To keep track of the random seed and create simulations
that can be reproduced using the same seed.  A definition
of the seed is not required when the MPS core is called
only for the generation of the lists. Therefore, in this
case the value of seed can be None, and the value of the
module variable \sphinxtitleref{seed\_default} is used.

\end{description}

\item[{Returns:}] \leavevmode
A \sphinxtitleref{*.in} file for the current simulation.

\end{description}

\begin{sphinxadmonition}{note}{Note:}
In order to increase the variability of the simulations, a new
random seed is generated for each simulation file that is
created if a \sphinxcode{\sphinxupquote{RandomState}} instance is provided with
the paramter \sphinxtitleref{seed}. Otherwise, the value of the global
variable \sphinxtitleref{seed\_default} is used.
\end{sphinxadmonition}

\end{fulllineitems}

\index{create\_lists() (in module s2Dcd.s2Dcd)@\spxentry{create\_lists()}\spxextra{in module s2Dcd.s2Dcd}}

\begin{fulllineitems}
\phantomsection\label{\detokenize{appendices:s2Dcd.s2Dcd.create_lists}}\pysiglinewithargsret{\sphinxcode{\sphinxupquote{s2Dcd.s2Dcd.}}\sphinxbfcode{\sphinxupquote{create\_lists}}}{\emph{simODS}, \emph{par\_template}, \emph{par\_Xnorm}, \emph{par\_Ynorm}, \emph{par\_Znorm}}{}
Create the lists which will be used by \sphinxstyleemphasis{Impala} for the simulation.

This preliminary step is required and allows use \sphinxstyleemphasis{Impala} to
compute the MPS list only once, before starting the simulation,
and therefore save computing resources.
\begin{description}
\item[{Parameters:}] \leavevmode\begin{description}
\item[{simODS: object of type \sphinxcode{\sphinxupquote{Grid}}}] \leavevmode
Contains all the info concerning the simulation domain.

\item[{par\_template: impala\_interface.Param}] \leavevmode
All the information contained in the \sphinxtitleref{template.in} file.

\item[{par\_Xnorm: object of type \sphinxcode{\sphinxupquote{impala\_interface.Param}}}] \leavevmode
Information about the \sphinxstyleemphasis{Impala} parameters normal to the
direction \sphinxstyleemphasis{x}.

\item[{par\_Ynorm: object of type \sphinxcode{\sphinxupquote{impala\_interface.Param}}}] \leavevmode
Information about the \sphinxstyleemphasis{Impala} parameters normal to the
direction \sphinxstyleemphasis{y}.

\item[{par\_Znorm: object of type \sphinxcode{\sphinxupquote{impala\_interface.Param}}}] \leavevmode
Information about the \sphinxstyleemphasis{Impala} parameters normal to the
direction \sphinxstyleemphasis{z}.

\end{description}

\item[{Returns: }] \leavevmode
A list (in binary format) for each multi\sphinxhyphen{}grid level and for
each simulation direction is created in the current directory
running \sphinxstyleemphasis{Impala}.

\end{description}

\end{fulllineitems}

\index{create\_seq() (in module s2Dcd.s2Dcd)@\spxentry{create\_seq()}\spxextra{in module s2Dcd.s2Dcd}}

\begin{fulllineitems}
\phantomsection\label{\detokenize{appendices:s2Dcd.s2Dcd.create_seq}}\pysiglinewithargsret{\sphinxcode{\sphinxupquote{s2Dcd.s2Dcd.}}\sphinxbfcode{\sphinxupquote{create\_seq}}}{\emph{simODS}, \emph{par\_Xnorm}, \emph{par\_Ynorm}, \emph{par\_Znorm}, \emph{pseudo3D=0}}{}
Creates a simple simulation sequence.
\begin{description}
\item[{Parameters:}] \leavevmode\begin{description}
\item[{simODS: grid definition}] \leavevmode
Information about the simulation grid.

\item[{par\_Xnorm, par\_Ynorm, par\_Znorm: \sphinxcode{\sphinxupquote{Param}}}] \leavevmode
Information related to the simulation directions.
See the class documentation for details.

\item[{pseudo3D: integer (default=0)}] \leavevmode
Set or not the “pseudo3D” simulation option when this
value is \textgreater{}0.

\end{description}

\item[{Returns:}] \leavevmode\begin{itemize}
\item {} 
A sequence of {\hyperref[\detokenize{appendices:s2Dcd.s2Dcd.SeqStep}]{\sphinxcrossref{\sphinxcode{\sphinxupquote{SeqStep}}}}} objects, with all the
information needed to run each simulation step.

\end{itemize}

\end{description}

\begin{sphinxadmonition}{note}{Note:}
This function provide the \sphinxstylestrong{basic} definition for a
simulation sequence when the simulation domain is quite
simple, like for example when it has a “box” shape (the sizes
of the simulation grid along the directions \sphinxstyleemphasis{x}, \sphinxstyleemphasis{y} and \sphinxstyleemphasis{z}
are comparable).  If the dimensions along the different axes
of your simulation domain are not comparable, you can for
example explicitly add some customized simulation steps to
improve the quality of the results. There is one example of
this in the examples directory.
\end{sphinxadmonition}

\end{fulllineitems}

\index{file\_name\_sec() (in module s2Dcd.s2Dcd)@\spxentry{file\_name\_sec()}\spxextra{in module s2Dcd.s2Dcd}}

\begin{fulllineitems}
\phantomsection\label{\detokenize{appendices:s2Dcd.s2Dcd.file_name_sec}}\pysiglinewithargsret{\sphinxcode{\sphinxupquote{s2Dcd.s2Dcd.}}\sphinxbfcode{\sphinxupquote{file\_name\_sec}}}{\emph{file\_name}, \emph{axe}}{}
Return the name of a file which will be used to store a section
of the input parameter \sphinxtitleref{file\_name}.
\begin{description}
\item[{Parameters:}] \leavevmode\begin{description}
\item[{file\_name: string}] \leavevmode
The name of the input file which should be sectioned

\item[{axe: string in (‘x’,’y’,’z’)}] \leavevmode
The axis perpendicular to the section.

\end{description}

\end{description}

\begin{sphinxadmonition}{note}{Note:}
It is supposed that the sectioned file lies in the current 
directory. Therefore, the path in the name of the input
file name will be dropped.
\end{sphinxadmonition}

\end{fulllineitems}

\index{matrioska\_interval() (in module s2Dcd.s2Dcd)@\spxentry{matrioska\_interval()}\spxextra{in module s2Dcd.s2Dcd}}

\begin{fulllineitems}
\phantomsection\label{\detokenize{appendices:s2Dcd.s2Dcd.matrioska_interval}}\pysiglinewithargsret{\sphinxcode{\sphinxupquote{s2Dcd.s2Dcd.}}\sphinxbfcode{\sphinxupquote{matrioska\_interval}}}{\emph{points\_nb}}{}
Computes a “spreaded” list of integers.

This function is useful to provide a simulation sequence along one
axis that allows, in priciple, to obtain as much as intersections
as possible along the other directions.
\begin{description}
\item[{Parameters:}] \leavevmode\begin{description}
\item[{points\_nb: integer}] \leavevmode
The number of points contained in the list.

\end{description}

\item[{Returns:}] \leavevmode
A list of integers containing a “matrioska sequence”.
If there is an error, returns \sphinxhyphen{}1.

\item[{Example:}] \leavevmode
A typical output of the function is:

\begin{sphinxVerbatim}[commandchars=\\\{\}]
\PYG{g+gp}{\PYGZgt{}\PYGZgt{}\PYGZgt{} }\PYG{n}{matrioska\PYGZus{}interval}\PYG{p}{(}\PYG{l+m+mi}{8}\PYG{p}{)}
\PYG{g+go}{[0, 7, 3, 1, 5, 2, 4, 6]}
\end{sphinxVerbatim}

\end{description}

\begin{sphinxadmonition}{note}{Note:}\begin{itemize}
\item {} 
The output sequence always start from 0.

\item {} 
The algorithm is probably not efficient, but for small
number, that is for sizes of simulations (\textless{}1000), it can be
OK.  Moreover, it should be called few times (max 3).

\end{itemize}
\end{sphinxadmonition}

\end{fulllineitems}

\index{numpy2hd4Impala() (in module s2Dcd.s2Dcd)@\spxentry{numpy2hd4Impala()}\spxextra{in module s2Dcd.s2Dcd}}

\begin{fulllineitems}
\phantomsection\label{\detokenize{appendices:s2Dcd.s2Dcd.numpy2hd4Impala}}\pysiglinewithargsret{\sphinxcode{\sphinxupquote{s2Dcd.s2Dcd.}}\sphinxbfcode{\sphinxupquote{numpy2hd4Impala}}}{\emph{simODS}, \emph{hd}, \emph{hd\_file}, \emph{seq=None}}{}
Given a numpy array, prints into an output file all the values in
the hard data file format of \sphinxstyleemphasis{Impala}.
\begin{description}
\item[{Parameters:}] \leavevmode\begin{description}
\item[{simODS: object of type \sphinxcode{\sphinxupquote{Grid}}}] \leavevmode
This object contains all the dimension required to define a
simulation domain.

\item[{hd: numpy array}] \leavevmode
A numpy array containing all the data which to be
converted into hard data. The array contains
\sphinxcode{\sphinxupquote{no\_data}} where there are not conditioning data.

\item[{hd\_file: string}] \leavevmode
Name of the file where the hard data will be printed.

\item[{seq: object type {\hyperref[\detokenize{appendices:s2Dcd.s2Dcd.SeqStep}]{\sphinxcrossref{\sphinxcode{\sphinxupquote{SeqStep}}}}}, optional}] \leavevmode
Some information about the current simulation step. 
If None, then all the \sphinxstyleemphasis{informed} content of the input
numpy file is saved.

\end{description}

\item[{Returns:}] \leavevmode
A file containing all the conditioning data required for a given
simulation step in the \sphinxstyleemphasis{Impala} hard data format.

\end{description}

\end{fulllineitems}

\index{numpy2hd4MPDS() (in module s2Dcd.s2Dcd)@\spxentry{numpy2hd4MPDS()}\spxextra{in module s2Dcd.s2Dcd}}

\begin{fulllineitems}
\phantomsection\label{\detokenize{appendices:s2Dcd.s2Dcd.numpy2hd4MPDS}}\pysiglinewithargsret{\sphinxcode{\sphinxupquote{s2Dcd.s2Dcd.}}\sphinxbfcode{\sphinxupquote{numpy2hd4MPDS}}}{\emph{simODS}, \emph{hd}, \emph{seq}}{}
Given a numpy array, prints into to a \sphinxtitleref{xyz\_data} file to be used
by the function \sphinxtitleref{gslibnumpy.numpy2gslib\_points} to create a file
with the conditioning points for the simulation for the MPS code
\sphinxtitleref{MPDS}.
\begin{description}
\item[{Parameters:}] \leavevmode\begin{description}
\item[{simODS: object of type \sphinxcode{\sphinxupquote{Grid}}}] \leavevmode
This object contains all the dimension required to define a
simulation domain.

\item[{hd: numpy array}] \leavevmode
A numpy array containing all the data which to be
converted into hard data. The array contains
\sphinxcode{\sphinxupquote{no\_data}} where there are not conditioning data.
seq: object type {\hyperref[\detokenize{appendices:s2Dcd.s2Dcd.SeqStep}]{\sphinxcrossref{\sphinxcode{\sphinxupquote{SeqStep}}}}} Some information
about the current simulation step.

\end{description}

\item[{Returns: }] \leavevmode
A “xyz\_data” file (a tuple of numpy arrays) containing the
information about the coordinates and the data.

\end{description}

\end{fulllineitems}

\index{print\_result() (in module s2Dcd.s2Dcd)@\spxentry{print\_result()}\spxextra{in module s2Dcd.s2Dcd}}

\begin{fulllineitems}
\phantomsection\label{\detokenize{appendices:s2Dcd.s2Dcd.print_result}}\pysiglinewithargsret{\sphinxcode{\sphinxupquote{s2Dcd.s2Dcd.}}\sphinxbfcode{\sphinxupquote{print\_result}}}{\emph{hd}, \emph{simODS}, \emph{file\_name}}{}
Print into an ouput file the simulation output.
\begin{description}
\item[{Parameters:}] \leavevmode\begin{description}
\item[{hd: numpy array}] \leavevmode
The numpy (3D) array containing all the simulated nodes.

\item[{simOSD: object of type \sphinxcode{\sphinxupquote{Grid}}}] \leavevmode
Contains the dimensions of the simulation domain.

\item[{file\_name: string}] \leavevmode
Name of the output file. The extension can be 
“vtk” or “gslib”.

\end{description}

\item[{Returns:}] \leavevmode
Create the output file.
If some error occurs, returns \sphinxhyphen{}1.

\end{description}

\end{fulllineitems}

\index{print\_sim\_info() (in module s2Dcd.s2Dcd)@\spxentry{print\_sim\_info()}\spxextra{in module s2Dcd.s2Dcd}}

\begin{fulllineitems}
\phantomsection\label{\detokenize{appendices:s2Dcd.s2Dcd.print_sim_info}}\pysiglinewithargsret{\sphinxcode{\sphinxupquote{s2Dcd.s2Dcd.}}\sphinxbfcode{\sphinxupquote{print\_sim\_info}}}{\emph{simODS}, \emph{par\_Xnorm}, \emph{par\_Ynorm}, \emph{par\_Znorm}}{}
Print some information about the parameters of the simulation.
\begin{description}
\item[{Parameters:}] \leavevmode\begin{description}
\item[{simODS: object of type \sphinxcode{\sphinxupquote{Grid}}}] \leavevmode
Information about the simulation grid.

\item[{par\_Xnorm: object of type \sphinxcode{\sphinxupquote{Param}} or \sphinxcode{\sphinxupquote{Param}}.}] \leavevmode
Information about the \sphinxtitleref{Impala} parameters normal to the
direction \sphinxstyleemphasis{x}.  (idem for \sphinxstyleemphasis{y} and \sphinxstyleemphasis{z}) and for \sphinxtitleref{MPDS}.

\end{description}

\end{description}

\end{fulllineitems}

\index{sim\_run() (in module s2Dcd.s2Dcd)@\spxentry{sim\_run()}\spxextra{in module s2Dcd.s2Dcd}}

\begin{fulllineitems}
\phantomsection\label{\detokenize{appendices:s2Dcd.s2Dcd.sim_run}}\pysiglinewithargsret{\sphinxcode{\sphinxupquote{s2Dcd.s2Dcd.}}\sphinxbfcode{\sphinxupquote{sim\_run}}}{\emph{seq\_steps}, \emph{step\_max}, \emph{hard\_data}, \emph{simODS}, \emph{in\_par}, \emph{seed}, \emph{res\_file\_root=\textquotesingle{}result\textquotesingle{}}, \emph{rcp\_lists=False}}{}
Run a \sphinxtitleref{s2Dcd} simulation.
\begin{description}
\item[{Parameters:}] \leavevmode\begin{description}
\item[{seq\_steps: list object of type {\hyperref[\detokenize{appendices:s2Dcd.s2Dcd.SeqStep}]{\sphinxcrossref{\sphinxcode{\sphinxupquote{SeqStep}}}}}}] \leavevmode
A list containing all the information of each simulation step.

\item[{step\_max: int}] \leavevmode
The maximum simulation step. Note that the simulation
domain can be often filled before this step.

\item[{hard\_data: numpy array}] \leavevmode
A numpy array containig all the simulation. The value
\sphinxcode{\sphinxupquote{no\_data}} is used in location not simulated.

\item[{simODS: object of type \sphinxcode{\sphinxupquote{Grid}}}] \leavevmode
Contains all the info concerning the simulation domain.

\item[{in\_par: string or a object of type \sphinxtitleref{ds\_interface.Param}}] \leavevmode
The name of the file containing the template of the input
parameters file for \sphinxtitleref{Impala} or the object containing all
the parameters required for a \sphinxtitleref{MPDS} simulation.

\item[{seed: instance of the class \sphinxcode{\sphinxupquote{RandomState}}}] \leavevmode
To keep track of the random seed and create simulations
that can be reproduced using the same seed.

\item[{res\_file\_root: string}] \leavevmode
The root name of the file containing the output
results. The right extension will be added once defined
the MPS simulation core.

\item[{rcp\_lists: flag (default False)}] \leavevmode
This flag is useful if one wants to re\sphinxhyphen{}compute the list
at each s2Dcd simulation step. When True it is a waste of time
if the TI along the same direction is always the same. It is
only useful when different training images are used for
the simuations along the same direction.

\end{description}

\item[{Returns:}] \leavevmode
A VTK or a GSLIB file containing the results of the simulation.
If the max number of interation is reached, a result file
containing “no data” values is printed out.

\end{description}

\end{fulllineitems}

\index{split\_segms() (in module s2Dcd.s2Dcd)@\spxentry{split\_segms()}\spxextra{in module s2Dcd.s2Dcd}}

\begin{fulllineitems}
\phantomsection\label{\detokenize{appendices:s2Dcd.s2Dcd.split_segms}}\pysiglinewithargsret{\sphinxcode{\sphinxupquote{s2Dcd.s2Dcd.}}\sphinxbfcode{\sphinxupquote{split\_segms}}}{\emph{segms}, \emph{mid\_list={[}{]}}}{}
A recursive function to print out a “matrioska\sphinxhyphen{}like” sequence of
integers.
\begin{description}
\item[{Parameters}] \leavevmode\begin{description}
\item[{segms}] \leavevmode{[}list of tuples of type (a,b){]}
A list containing a number of tuples representing a segment.

\item[{mid\_list: list of integers, optional}] \leavevmode
The list of integers that represents the required output.
By default this should be called without this argument.

\end{description}

\item[{Returns}] \leavevmode
A list containing all the integers in the mixed “matrioska” order.

\item[{Example}] \leavevmode
\begin{sphinxVerbatim}[commandchars=\\\{\}]
\PYG{g+gp}{\PYGZgt{}\PYGZgt{}\PYGZgt{} }\PYG{n}{matrioska\PYGZus{}interval}\PYG{p}{(}\PYG{p}{[}\PYG{p}{(}\PYG{l+m+mi}{11}\PYG{p}{,}\PYG{l+m+mi}{26}\PYG{p}{)}\PYG{p}{]}\PYG{p}{)}
\PYG{g+go}{[11, 26, 18, 14, 22, 12, 16, 20, 24, 13, 15, 17, 19, 21, 23, 25]}
\end{sphinxVerbatim}

\end{description}

\begin{sphinxadmonition}{note}{Note:}
It is supposed that the coordinates of the segments are all
integers.
\end{sphinxadmonition}

\end{fulllineitems}



\section{The \sphinxstyleliteralintitle{\sphinxupquote{s2Dcd.deesse}} module}
\label{\detokenize{appendices:module-s2Dcd.deesse}}\label{\detokenize{appendices:the-s2dcd-deesse-module}}\index{s2Dcd.deesse (module)@\spxentry{s2Dcd.deesse}\spxextra{module}}\begin{quote}\begin{description}
\item[{license}] \leavevmode
This file is part of s2Dcd.

s2Dcd is free software: you can redistribute it and/or modify
it under the terms of the GNU General Public License as published by
the Free Software Foundation, either version 3 of the License, or
(at your option) any later version.

s2Dcd is distributed in the hope that it will be useful,
but WITHOUT ANY WARRANTY; without even the implied warranty of
MERCHANTABILITY or FITNESS FOR A PARTICULAR PURPOSE.  See the
GNU General Public License for more details.

You should have received a copy of the GNU General Public License
along with s2Dcd.  If not, see \textless{}\sphinxurl{https://www.gnu.org/licenses/}\textgreater{}.

\item[{Purpose}] \leavevmode
A collection of classes and functions to interact with the
Multiple Point Direct Sampling code (DeeSse).

\item[{File name}] \leavevmode
\sphinxcode{\sphinxupquote{deesse.py}}

\item[{Version}] \leavevmode\begin{description}
\item[{0.7 , 2020\sphinxhyphen{}01\sphinxhyphen{}20 :}] \leavevmode\begin{itemize}
\item {} 
Some clean up before the upload to Github.

\end{itemize}

\item[{0.6 , 2017\sphinxhyphen{}12\sphinxhyphen{}15 :}] \leavevmode\begin{itemize}
\item {} 
Adapted to the 2017 vesion of the code (includes “Pyramids” option)

\item {} 
Partially takes the comments for the .in file from an
external .json file.

\end{itemize}

\item[{0.5 , 2015\sphinxhyphen{}10\sphinxhyphen{}15 :}] \leavevmode\begin{itemize}
\item {} 
Extending to deal with multiple variables simulation.

\end{itemize}

\item[{0.4 , 2013\sphinxhyphen{}06\sphinxhyphen{}17 :}] \leavevmode\begin{itemize}
\item {} 
Some small modifications to make the setting of the default values
more coherent.

\end{itemize}

\item[{0.3 , 2013\sphinxhyphen{}03\sphinxhyphen{}13 :}] \leavevmode\begin{itemize}
\item {} 
Moved to Python3 using \sphinxtitleref{2to3}.

\end{itemize}

\item[{0.2 , 2012\sphinxhyphen{}07\sphinxhyphen{}30 :}] \leavevmode\begin{itemize}
\item {} 
Moved the function \sphinxtitleref{skip\_ccomments} to the module \sphinxtitleref{utili.py}.

\item {} 
Last version before the movement to Python3.

\end{itemize}

\item[{0.1 , 2012\sphinxhyphen{}03\sphinxhyphen{}30 :}] \leavevmode
First version.

\end{description}

\item[{Authors}] \leavevmode
Alessandro Comunian

\end{description}\end{quote}

\begin{sphinxadmonition}{note}{Note:}\begin{itemize}
\item {} 
A \sphinxtitleref{deesse\_file\_in\_text.json} is required and contains all comment text
that put in the deesse \sphinxtitleref{.in} file. To create easily a \sphinxtitleref{.json} file that
respects the intendation and the new lines of the original \sphinxtitleref{.in} file,
you can use something like:

\begin{sphinxVerbatim}[commandchars=\\\{\}]
\PYG{n}{gawk} \PYG{l+s+s1}{\PYGZsq{}}\PYG{l+s+s1}{\PYGZdl{}1=\PYGZdl{}0}\PYG{l+s+s1}{\PYGZsq{}} \PYG{n}{ORS}\PYG{o}{=}\PYG{l+s+s1}{\PYGZsq{}}\PYG{l+s+se}{\PYGZbs{}n}\PYG{l+s+s1}{\PYGZsq{}} \PYG{n}{file}\PYG{o}{.}\PYG{o+ow}{in}
\end{sphinxVerbatim}

to replace all the new lines by n, at the same time keeping the
original indentation of the file.

\end{itemize}
\end{sphinxadmonition}

\begin{sphinxadmonition}{warning}{Warning:}\begin{itemize}
\item {} 
The coverage and the support of the simulation options provided
by DS simulation engine is only partial.

\end{itemize}
\end{sphinxadmonition}
\index{Param (class in s2Dcd.deesse)@\spxentry{Param}\spxextra{class in s2Dcd.deesse}}

\begin{fulllineitems}
\phantomsection\label{\detokenize{appendices:s2Dcd.deesse.Param}}\pysiglinewithargsret{\sphinxbfcode{\sphinxupquote{class }}\sphinxcode{\sphinxupquote{s2Dcd.deesse.}}\sphinxbfcode{\sphinxupquote{Param}}}{\emph{file\_name=None, gridin=\textless{}s2Dcd.grid.Grid object\textgreater{}, var\_nb=1, vars={[}\textless{}s2Dcd.deesse.VarInfo object\textgreater{}{]}, out\_set=\textquotesingle{}OUTPUT\_SIM\_ALL\_IN\_ONE\_FILE\textquotesingle{}, out\_file=\textquotesingle{}test\_simul.gslib\textquotesingle{}, out\_report=True, out\_report\_file=\textquotesingle{}test\_report.txt\textquotesingle{}, ti\_nb=1, data\_img\_nb=0, data\_img\_files={[}{]}, data\_pointset\_nb=0, data\_pointset\_files={[}{]}, mask=False, homot\_usage=0, rot\_usage=0, cond\_data\_tol=0.05, norm\_type=\textquotesingle{}NORMALIZING\_LINEAR\textquotesingle{}, sim\_type=\textquotesingle{}SIM\_ONE\_BY\_ONE\textquotesingle{}, sim\_path=\textquotesingle{}PATH\_RANDOM\textquotesingle{}, tol=0.0, post\_proc\_path\_max=1, post\_proc\_par=\textquotesingle{}POST\_PROCESSING\_PARAMETERS\_DEFAULT\textquotesingle{}, pyramids=0, seed=444, seed\_inc=1, real\_nb=1}}{}
A class containing all the parameters contained in the parameters file.
\index{print\_blockdata() (s2Dcd.deesse.Param method)@\spxentry{print\_blockdata()}\spxextra{s2Dcd.deesse.Param method}}

\begin{fulllineitems}
\phantomsection\label{\detokenize{appendices:s2Dcd.deesse.Param.print_blockdata}}\pysiglinewithargsret{\sphinxbfcode{\sphinxupquote{print\_blockdata}}}{}{}
Print info related to the BLOCK\_DATA option

\end{fulllineitems}

\index{print\_blockdata\_intro() (s2Dcd.deesse.Param method)@\spxentry{print\_blockdata\_intro()}\spxextra{s2Dcd.deesse.Param method}}

\begin{fulllineitems}
\phantomsection\label{\detokenize{appendices:s2Dcd.deesse.Param.print_blockdata_intro}}\pysiglinewithargsret{\sphinxbfcode{\sphinxupquote{print\_blockdata\_intro}}}{}{}
Print some introductory text to the 
parameters related to the block data option

\begin{sphinxadmonition}{warning}{Warning:}
Probability constraints are still not implemented in 
this version of the DeeSse interface.
\end{sphinxadmonition}

\end{fulllineitems}

\index{print\_consi() (s2Dcd.deesse.Param method)@\spxentry{print\_consi()}\spxextra{s2Dcd.deesse.Param method}}

\begin{fulllineitems}
\phantomsection\label{\detokenize{appendices:s2Dcd.deesse.Param.print_consi}}\pysiglinewithargsret{\sphinxbfcode{\sphinxupquote{print\_consi}}}{}{}
Print info about the consistency

\end{fulllineitems}

\index{print\_difile() (s2Dcd.deesse.Param method)@\spxentry{print\_difile()}\spxextra{s2Dcd.deesse.Param method}}

\begin{fulllineitems}
\phantomsection\label{\detokenize{appendices:s2Dcd.deesse.Param.print_difile}}\pysiglinewithargsret{\sphinxbfcode{\sphinxupquote{print\_difile}}}{}{}
Print the content of the data image file

\end{fulllineitems}

\index{print\_distthr() (s2Dcd.deesse.Param method)@\spxentry{print\_distthr()}\spxextra{s2Dcd.deesse.Param method}}

\begin{fulllineitems}
\phantomsection\label{\detokenize{appendices:s2Dcd.deesse.Param.print_distthr}}\pysiglinewithargsret{\sphinxbfcode{\sphinxupquote{print\_distthr}}}{}{}
Print info about the distance threshold

\end{fulllineitems}

\index{print\_disttype() (s2Dcd.deesse.Param method)@\spxentry{print\_disttype()}\spxextra{s2Dcd.deesse.Param method}}

\begin{fulllineitems}
\phantomsection\label{\detokenize{appendices:s2Dcd.deesse.Param.print_disttype}}\pysiglinewithargsret{\sphinxbfcode{\sphinxupquote{print\_disttype}}}{}{}
Print info related to the distance type

\end{fulllineitems}

\index{print\_dpset() (s2Dcd.deesse.Param method)@\spxentry{print\_dpset()}\spxextra{s2Dcd.deesse.Param method}}

\begin{fulllineitems}
\phantomsection\label{\detokenize{appendices:s2Dcd.deesse.Param.print_dpset}}\pysiglinewithargsret{\sphinxbfcode{\sphinxupquote{print\_dpset}}}{}{}
Print info related to the data point set

\end{fulllineitems}

\index{print\_file() (s2Dcd.deesse.Param method)@\spxentry{print\_file()}\spxextra{s2Dcd.deesse.Param method}}

\begin{fulllineitems}
\phantomsection\label{\detokenize{appendices:s2Dcd.deesse.Param.print_file}}\pysiglinewithargsret{\sphinxbfcode{\sphinxupquote{print\_file}}}{\emph{file\_name}}{}
Print the content of the parameters object into a file.
\begin{description}
\item[{Parameters:}] \leavevmode\begin{description}
\item[{file\_name: string}] \leavevmode
Name of the file where to print.

\end{description}

\end{description}

\end{fulllineitems}

\index{print\_homo() (s2Dcd.deesse.Param method)@\spxentry{print\_homo()}\spxextra{s2Dcd.deesse.Param method}}

\begin{fulllineitems}
\phantomsection\label{\detokenize{appendices:s2Dcd.deesse.Param.print_homo}}\pysiglinewithargsret{\sphinxbfcode{\sphinxupquote{print\_homo}}}{}{}
Print info about the homothety

\end{fulllineitems}

\index{print\_maskimage() (s2Dcd.deesse.Param method)@\spxentry{print\_maskimage()}\spxextra{s2Dcd.deesse.Param method}}

\begin{fulllineitems}
\phantomsection\label{\detokenize{appendices:s2Dcd.deesse.Param.print_maskimage}}\pysiglinewithargsret{\sphinxbfcode{\sphinxupquote{print\_maskimage}}}{}{}
Print infor related to the mask image

\end{fulllineitems}

\index{print\_max\_dens() (s2Dcd.deesse.Param method)@\spxentry{print\_max\_dens()}\spxextra{s2Dcd.deesse.Param method}}

\begin{fulllineitems}
\phantomsection\label{\detokenize{appendices:s2Dcd.deesse.Param.print_max_dens}}\pysiglinewithargsret{\sphinxbfcode{\sphinxupquote{print\_max\_dens}}}{}{}
Print info about the max density of nodes

\end{fulllineitems}

\index{print\_max\_nod() (s2Dcd.deesse.Param method)@\spxentry{print\_max\_nod()}\spxextra{s2Dcd.deesse.Param method}}

\begin{fulllineitems}
\phantomsection\label{\detokenize{appendices:s2Dcd.deesse.Param.print_max_nod}}\pysiglinewithargsret{\sphinxbfcode{\sphinxupquote{print\_max\_nod}}}{}{}
Print the max number of neighborihg nodes

\end{fulllineitems}

\index{print\_maxscan() (s2Dcd.deesse.Param method)@\spxentry{print\_maxscan()}\spxextra{s2Dcd.deesse.Param method}}

\begin{fulllineitems}
\phantomsection\label{\detokenize{appendices:s2Dcd.deesse.Param.print_maxscan}}\pysiglinewithargsret{\sphinxbfcode{\sphinxupquote{print\_maxscan}}}{}{}
Print info about the percentage of image to scan

\end{fulllineitems}

\index{print\_neigh\_intro() (s2Dcd.deesse.Param method)@\spxentry{print\_neigh\_intro()}\spxextra{s2Dcd.deesse.Param method}}

\begin{fulllineitems}
\phantomsection\label{\detokenize{appendices:s2Dcd.deesse.Param.print_neigh_intro}}\pysiglinewithargsret{\sphinxbfcode{\sphinxupquote{print\_neigh\_intro}}}{}{}
Print some introductory text to the 
parameters related to the search neighborhood

\end{fulllineitems}

\index{print\_neighs() (s2Dcd.deesse.Param method)@\spxentry{print\_neighs()}\spxextra{s2Dcd.deesse.Param method}}

\begin{fulllineitems}
\phantomsection\label{\detokenize{appendices:s2Dcd.deesse.Param.print_neighs}}\pysiglinewithargsret{\sphinxbfcode{\sphinxupquote{print\_neighs}}}{}{}
Print info related to the search neighborhood

\end{fulllineitems}

\index{print\_norm() (s2Dcd.deesse.Param method)@\spxentry{print\_norm()}\spxextra{s2Dcd.deesse.Param method}}

\begin{fulllineitems}
\phantomsection\label{\detokenize{appendices:s2Dcd.deesse.Param.print_norm}}\pysiglinewithargsret{\sphinxbfcode{\sphinxupquote{print\_norm}}}{}{}
Print info about the normalization

\end{fulllineitems}

\index{print\_out\_report() (s2Dcd.deesse.Param method)@\spxentry{print\_out\_report()}\spxextra{s2Dcd.deesse.Param method}}

\begin{fulllineitems}
\phantomsection\label{\detokenize{appendices:s2Dcd.deesse.Param.print_out_report}}\pysiglinewithargsret{\sphinxbfcode{\sphinxupquote{print\_out\_report}}}{}{}
Print the output report

\end{fulllineitems}

\index{print\_out\_set() (s2Dcd.deesse.Param method)@\spxentry{print\_out\_set()}\spxextra{s2Dcd.deesse.Param method}}

\begin{fulllineitems}
\phantomsection\label{\detokenize{appendices:s2Dcd.deesse.Param.print_out_set}}\pysiglinewithargsret{\sphinxbfcode{\sphinxupquote{print\_out\_set}}}{}{}
Print the output settings

\end{fulllineitems}

\index{print\_post() (s2Dcd.deesse.Param method)@\spxentry{print\_post()}\spxextra{s2Dcd.deesse.Param method}}

\begin{fulllineitems}
\phantomsection\label{\detokenize{appendices:s2Dcd.deesse.Param.print_post}}\pysiglinewithargsret{\sphinxbfcode{\sphinxupquote{print\_post}}}{}{}
Print info related to the post\sphinxhyphen{}processing procedure

\end{fulllineitems}

\index{print\_proconst() (s2Dcd.deesse.Param method)@\spxentry{print\_proconst()}\spxextra{s2Dcd.deesse.Param method}}

\begin{fulllineitems}
\phantomsection\label{\detokenize{appendices:s2Dcd.deesse.Param.print_proconst}}\pysiglinewithargsret{\sphinxbfcode{\sphinxupquote{print\_proconst}}}{}{}
Print info related to the probability constraints

\end{fulllineitems}

\index{print\_proconst\_intro() (s2Dcd.deesse.Param method)@\spxentry{print\_proconst\_intro()}\spxextra{s2Dcd.deesse.Param method}}

\begin{fulllineitems}
\phantomsection\label{\detokenize{appendices:s2Dcd.deesse.Param.print_proconst_intro}}\pysiglinewithargsret{\sphinxbfcode{\sphinxupquote{print\_proconst\_intro}}}{}{}
Print some introductory text to the 
parameters related to the probability constraints

\begin{sphinxadmonition}{warning}{Warning:}
Probability constraints are still not implemented in 
this version of the DeeSse interface.
\end{sphinxadmonition}

\end{fulllineitems}

\index{print\_pyramids() (s2Dcd.deesse.Param method)@\spxentry{print\_pyramids()}\spxextra{s2Dcd.deesse.Param method}}

\begin{fulllineitems}
\phantomsection\label{\detokenize{appendices:s2Dcd.deesse.Param.print_pyramids}}\pysiglinewithargsret{\sphinxbfcode{\sphinxupquote{print\_pyramids}}}{}{}
Print info related to the pyramids procedure

\end{fulllineitems}

\index{print\_realnb() (s2Dcd.deesse.Param method)@\spxentry{print\_realnb()}\spxextra{s2Dcd.deesse.Param method}}

\begin{fulllineitems}
\phantomsection\label{\detokenize{appendices:s2Dcd.deesse.Param.print_realnb}}\pysiglinewithargsret{\sphinxbfcode{\sphinxupquote{print\_realnb}}}{}{}
Print the number of realizations

\end{fulllineitems}

\index{print\_reldist() (s2Dcd.deesse.Param method)@\spxentry{print\_reldist()}\spxextra{s2Dcd.deesse.Param method}}

\begin{fulllineitems}
\phantomsection\label{\detokenize{appendices:s2Dcd.deesse.Param.print_reldist}}\pysiglinewithargsret{\sphinxbfcode{\sphinxupquote{print\_reldist}}}{}{}
Print info relative to the relative distance

\end{fulllineitems}

\index{print\_rot() (s2Dcd.deesse.Param method)@\spxentry{print\_rot()}\spxextra{s2Dcd.deesse.Param method}}

\begin{fulllineitems}
\phantomsection\label{\detokenize{appendices:s2Dcd.deesse.Param.print_rot}}\pysiglinewithargsret{\sphinxbfcode{\sphinxupquote{print\_rot}}}{}{}
Print info about the rotation

\end{fulllineitems}

\index{print\_seed() (s2Dcd.deesse.Param method)@\spxentry{print\_seed()}\spxextra{s2Dcd.deesse.Param method}}

\begin{fulllineitems}
\phantomsection\label{\detokenize{appendices:s2Dcd.deesse.Param.print_seed}}\pysiglinewithargsret{\sphinxbfcode{\sphinxupquote{print\_seed}}}{}{}
Print info about the seed

\end{fulllineitems}

\index{print\_sim\_grid() (s2Dcd.deesse.Param method)@\spxentry{print\_sim\_grid()}\spxextra{s2Dcd.deesse.Param method}}

\begin{fulllineitems}
\phantomsection\label{\detokenize{appendices:s2Dcd.deesse.Param.print_sim_grid}}\pysiglinewithargsret{\sphinxbfcode{\sphinxupquote{print\_sim\_grid}}}{}{}
Print the simulation grid info

\end{fulllineitems}

\index{print\_sim\_var() (s2Dcd.deesse.Param method)@\spxentry{print\_sim\_var()}\spxextra{s2Dcd.deesse.Param method}}

\begin{fulllineitems}
\phantomsection\label{\detokenize{appendices:s2Dcd.deesse.Param.print_sim_var}}\pysiglinewithargsret{\sphinxbfcode{\sphinxupquote{print\_sim\_var}}}{}{}
Print the simulation variables info

\end{fulllineitems}

\index{print\_simpath() (s2Dcd.deesse.Param method)@\spxentry{print\_simpath()}\spxextra{s2Dcd.deesse.Param method}}

\begin{fulllineitems}
\phantomsection\label{\detokenize{appendices:s2Dcd.deesse.Param.print_simpath}}\pysiglinewithargsret{\sphinxbfcode{\sphinxupquote{print\_simpath}}}{}{}
Print info about the simulation and the path

\end{fulllineitems}

\index{print\_ti() (s2Dcd.deesse.Param method)@\spxentry{print\_ti()}\spxextra{s2Dcd.deesse.Param method}}

\begin{fulllineitems}
\phantomsection\label{\detokenize{appendices:s2Dcd.deesse.Param.print_ti}}\pysiglinewithargsret{\sphinxbfcode{\sphinxupquote{print\_ti}}}{}{}
Print the TI info

\end{fulllineitems}

\index{print\_tol() (s2Dcd.deesse.Param method)@\spxentry{print\_tol()}\spxextra{s2Dcd.deesse.Param method}}

\begin{fulllineitems}
\phantomsection\label{\detokenize{appendices:s2Dcd.deesse.Param.print_tol}}\pysiglinewithargsret{\sphinxbfcode{\sphinxupquote{print\_tol}}}{}{}
Print info about the tolerance

\end{fulllineitems}

\index{print\_weight() (s2Dcd.deesse.Param method)@\spxentry{print\_weight()}\spxextra{s2Dcd.deesse.Param method}}

\begin{fulllineitems}
\phantomsection\label{\detokenize{appendices:s2Dcd.deesse.Param.print_weight}}\pysiglinewithargsret{\sphinxbfcode{\sphinxupquote{print\_weight}}}{}{}
Print info about the weighting factors

\end{fulllineitems}


\end{fulllineitems}

\index{TiInfo (class in s2Dcd.deesse)@\spxentry{TiInfo}\spxextra{class in s2Dcd.deesse}}

\begin{fulllineitems}
\phantomsection\label{\detokenize{appendices:s2Dcd.deesse.TiInfo}}\pysiglinewithargsret{\sphinxbfcode{\sphinxupquote{class }}\sphinxcode{\sphinxupquote{s2Dcd.deesse.}}\sphinxbfcode{\sphinxupquote{TiInfo}}}{\emph{file\_name=\textquotesingle{}ti.gslib\textquotesingle{}}, \emph{max\_scan=0.3}}{}
A class to support the class ParamMPDS. Contains all the information
related to one training image.
\begin{description}
\item[{Main attributes:}] \leavevmode\begin{description}
\item[{file\_name: string}] \leavevmode
Name of the file containing the training image

\item[{max\_scan: float}] \leavevmode
Fraction of the TI to scan

\end{description}

\end{description}

\end{fulllineitems}

\index{VarInfo (class in s2Dcd.deesse)@\spxentry{VarInfo}\spxextra{class in s2Dcd.deesse}}

\begin{fulllineitems}
\phantomsection\label{\detokenize{appendices:s2Dcd.deesse.VarInfo}}\pysiglinewithargsret{\sphinxbfcode{\sphinxupquote{class }}\sphinxcode{\sphinxupquote{s2Dcd.deesse.}}\sphinxbfcode{\sphinxupquote{VarInfo}}}{\emph{name=\textquotesingle{}facies\textquotesingle{}}, \emph{out\_flag=True}, \emph{fmt=\textquotesingle{}\%10.5E\textquotesingle{}}, \emph{search\_r\_x=120.0}, \emph{search\_r\_y=120.0}, \emph{search\_r\_z=0.0}, \emph{search\_anis\_x=1.0}, \emph{search\_anis\_y=1.0}, \emph{search\_anis\_z=1.0}, \emph{search\_ang\_az=0.0}, \emph{search\_ang\_dp=0.0}, \emph{search\_ang\_pl=0.0}, \emph{search\_pow=0.0}, \emph{max\_nb\_neigh=20}, \emph{max\_dens\_neigh=1.0}, \emph{rel\_dist\_flag=0}, \emph{dist\_type=0}, \emph{weight=1.0}, \emph{dist\_thre=0.01}, \emph{prob\_constr=0}, \emph{block\_data=0}}{}
A class to support the class ParamMPDS. It tries to regroup all
the information about a variable which have to be simulated.

\end{fulllineitems}



\section{The \sphinxstyleliteralintitle{\sphinxupquote{s2Dcd.utili}} module}
\label{\detokenize{appendices:module-s2Dcd.utili}}\label{\detokenize{appendices:the-s2dcd-utili-module}}\index{s2Dcd.utili (module)@\spxentry{s2Dcd.utili}\spxextra{module}}\begin{quote}\begin{description}
\item[{license}] \leavevmode
This file is part of s2Dcd.

s2Dcd is free software: you can redistribute it and/or modify
it under the terms of the GNU General Public License as published by
the Free Software Foundation, either version 3 of the License, or
(at your option) any later version.

s2Dcd is distributed in the hope that it will be useful,
but WITHOUT ANY WARRANTY; without even the implied warranty of
MERCHANTABILITY or FITNESS FOR A PARTICULAR PURPOSE.  See the
GNU General Public License for more details.

You should have received a copy of the GNU General Public License
along with s2Dcd.  If not, see \textless{}\sphinxurl{https://www.gnu.org/licenses/}\textgreater{}.

\item[{this file}] \leavevmode
\sphinxtitleref{utili.py}

\item[{Purpose}] \leavevmode
A collection of small utility functions

\item[{Version}] \leavevmode\begin{itemize}
\item {} \begin{description}
\item[{0.6 , 2017\sphinxhyphen{}12\sphinxhyphen{}15 :}] \leavevmode\begin{itemize}
\item {} 
Improved the way computing time is computed by using
\sphinxcode{\sphinxupquote{time.perf\_counter()}}

\end{itemize}

\end{description}

\item {} \begin{description}
\item[{0.5 , 2015\sphinxhyphen{}12\sphinxhyphen{}03 :}] \leavevmode\begin{itemize}
\item {} 
Some minor cleanup

\end{itemize}

\end{description}

\item {} \begin{description}
\item[{0.4 , 2013\sphinxhyphen{}03\sphinxhyphen{}06 :}] \leavevmode\begin{itemize}
\item {} 
Converted to Python3.X

\end{itemize}

\end{description}

\item {} \begin{description}
\item[{0.3 , 2013\sphinxhyphen{}03\sphinxhyphen{}06 :}] \leavevmode\begin{itemize}
\item {} 
Last version before the conversion to python3.X

\item {} 
Added a function to skip the C\sphinxhyphen{}like comments.

\end{itemize}

\end{description}

\item {} \begin{description}
\item[{0.2 , 2012\sphinxhyphen{}05\sphinxhyphen{}02 :}] \leavevmode\begin{itemize}
\item {} 
Added a function usefult for the output print…

\end{itemize}

\end{description}

\item {} \begin{description}
\item[{0.1 , 2012\sphinxhyphen{}03\sphinxhyphen{}09 :}] \leavevmode\begin{itemize}
\item {} 
First version.

\end{itemize}

\end{description}

\end{itemize}

\item[{Authors}] \leavevmode
Alessandro Comunian

\end{description}\end{quote}
\index{add\_file\_id() (in module s2Dcd.utili)@\spxentry{add\_file\_id()}\spxextra{in module s2Dcd.utili}}

\begin{fulllineitems}
\phantomsection\label{\detokenize{appendices:s2Dcd.utili.add_file_id}}\pysiglinewithargsret{\sphinxcode{\sphinxupquote{s2Dcd.utili.}}\sphinxbfcode{\sphinxupquote{add\_file\_id}}}{\emph{file\_name}, \emph{file\_id}}{}
Add a file ID to a file name, given an int ID.
\begin{description}
\item[{Parameters:}] \leavevmode\begin{description}
\item[{file\_name: string}] \leavevmode
The name of the file (with extension…)

\item[{file\_id: int}] \leavevmode
The ID which is associated to the file name

\end{description}

\end{description}

\end{fulllineitems}

\index{dtype\_fmt() (in module s2Dcd.utili)@\spxentry{dtype\_fmt()}\spxextra{in module s2Dcd.utili}}

\begin{fulllineitems}
\phantomsection\label{\detokenize{appendices:s2Dcd.utili.dtype_fmt}}\pysiglinewithargsret{\sphinxcode{\sphinxupquote{s2Dcd.utili.}}\sphinxbfcode{\sphinxupquote{dtype\_fmt}}}{\emph{x}}{}
Returns a printing format according to the type of the input numpy
array.
\begin{description}
\item[{Parameters:}] \leavevmode
x: numpy array

\item[{Returns:}] \leavevmode
A string that defines the output format that should be used for the
formatted output, that is ‘\%d’ is the case of an integer, and
‘\%.4e’ in case of a float.
In case of an error, returs \sphinxhyphen{}1.

\end{description}

\end{fulllineitems}

\index{print\_start() (in module s2Dcd.utili)@\spxentry{print\_start()}\spxextra{in module s2Dcd.utili}}

\begin{fulllineitems}
\phantomsection\label{\detokenize{appendices:s2Dcd.utili.print_start}}\pysiglinewithargsret{\sphinxcode{\sphinxupquote{s2Dcd.utili.}}\sphinxbfcode{\sphinxupquote{print\_start}}}{\emph{title=None}}{}
Print the “begin” header and start counting time…
\begin{description}
\item[{Parameters:}] \leavevmode\begin{description}
\item[{title: string, optional}] \leavevmode
A message to be printed in a “formatted stype”. If a
string is not provided, then use the name of the calling
script.

\end{description}

\item[{Returns:}] \leavevmode
The output of the function \sphinxtitleref{time.clock()}.

\end{description}

\end{fulllineitems}

\index{print\_stop() (in module s2Dcd.utili)@\spxentry{print\_stop()}\spxextra{in module s2Dcd.utili}}

\begin{fulllineitems}
\phantomsection\label{\detokenize{appendices:s2Dcd.utili.print_stop}}\pysiglinewithargsret{\sphinxcode{\sphinxupquote{s2Dcd.utili.}}\sphinxbfcode{\sphinxupquote{print\_stop}}}{\emph{time\_start=None}}{}
Print the “end” message and the computing time.
\begin{description}
\item[{Parameters:}] \leavevmode\begin{description}
\item[{time\_start: ouput of the function {\hyperref[\detokenize{appendices:s2Dcd.utili.print_start}]{\sphinxcrossref{\sphinxcode{\sphinxupquote{print\_start()}}}}}, optional}] \leavevmode
The function should be called in conjunction with the
function \sphinxtitleref{print\_start}. If a value for time\_start is not
provided, then simply print out the current time and the
stop message.

\end{description}

\item[{Returns:}] \leavevmode
Print out the computing time

\end{description}

\end{fulllineitems}

\index{progress\_bar() (in module s2Dcd.utili)@\spxentry{progress\_bar()}\spxextra{in module s2Dcd.utili}}

\begin{fulllineitems}
\phantomsection\label{\detokenize{appendices:s2Dcd.utili.progress_bar}}\pysiglinewithargsret{\sphinxcode{\sphinxupquote{s2Dcd.utili.}}\sphinxbfcode{\sphinxupquote{progress\_bar}}}{\emph{i}, \emph{nb\_steps}}{}
A simple progress bar
\begin{description}
\item[{Parameters:}] \leavevmode\begin{description}
\item[{i: integer}] \leavevmode
The current index

\item[{nb\_steps:}] \leavevmode
The total number of steps

\end{description}

\end{description}

\end{fulllineitems}

\index{progress\_bar\_end() (in module s2Dcd.utili)@\spxentry{progress\_bar\_end()}\spxextra{in module s2Dcd.utili}}

\begin{fulllineitems}
\phantomsection\label{\detokenize{appendices:s2Dcd.utili.progress_bar_end}}\pysiglinewithargsret{\sphinxcode{\sphinxupquote{s2Dcd.utili.}}\sphinxbfcode{\sphinxupquote{progress\_bar\_end}}}{}{}
Print the end of the progress bar

\end{fulllineitems}

\index{savepdf() (in module s2Dcd.utili)@\spxentry{savepdf()}\spxextra{in module s2Dcd.utili}}

\begin{fulllineitems}
\phantomsection\label{\detokenize{appendices:s2Dcd.utili.savepdf}}\pysiglinewithargsret{\sphinxcode{\sphinxupquote{s2Dcd.utili.}}\sphinxbfcode{\sphinxupquote{savepdf}}}{\emph{filename}, \emph{title=\textquotesingle{}\textquotesingle{}}}{}
Save a matplotlib figure as PDF and set up some useful information as
metadata in the PDF file.
\begin{description}
\item[{Parameters:}] \leavevmode\begin{description}
\item[{filename}] \leavevmode{[}string{]}
The name of the PDF file.

\item[{title}] \leavevmode{[}string, optional{]}
The title of the PDF document.

\end{description}

\item[{Returns:}] \leavevmode
A PDF file containing the figure with the selected metadata.

\end{description}

\end{fulllineitems}

\index{skip\_ccomments() (in module s2Dcd.utili)@\spxentry{skip\_ccomments()}\spxextra{in module s2Dcd.utili}}

\begin{fulllineitems}
\phantomsection\label{\detokenize{appendices:s2Dcd.utili.skip_ccomments}}\pysiglinewithargsret{\sphinxcode{\sphinxupquote{s2Dcd.utili.}}\sphinxbfcode{\sphinxupquote{skip\_ccomments}}}{\emph{par\_file}}{}
Read the connection to a file and extract a list containing all
the fields, excluding line commenten in a \sphinxstyleemphasis{C\sphinxhyphen{}like} style, that is
starting with “//” or enclosed by “/\sphinxstyleemphasis{…}/”.
\begin{description}
\item[{Parameters}] \leavevmode\begin{description}
\item[{file: file object}] \leavevmode
The file to be read.

\end{description}

\item[{Returns}] \leavevmode
A list containing all the strings (the parameters), separated
by a space.  Of course, the structure of the list depends on
the structure of the paramters file.

\end{description}

\begin{sphinxadmonition}{note}{Note:}
For the moment only the space characters are considered as field
separators. All the fields in the list are returned as strings,
therefore you will have to “cast” them according to your needs.
\end{sphinxadmonition}

\begin{sphinxadmonition}{warning}{Warning:}
The comments at the end of a line containing parameters are
not deleted. Therefore, used should be aware about what they
are reading on each line…
\end{sphinxadmonition}

\end{fulllineitems}



\section{The \sphinxstyleliteralintitle{\sphinxupquote{s2Dcd.grid}} module}
\label{\detokenize{appendices:module-s2Dcd.grid}}\label{\detokenize{appendices:the-s2dcd-grid-module}}\index{s2Dcd.grid (module)@\spxentry{s2Dcd.grid}\spxextra{module}}\begin{quote}\begin{description}
\item[{license}] \leavevmode
This file is part of s2Dcd.

s2Dcd is free software: you can redistribute it and/or modify
it under the terms of the GNU General Public License as published by
the Free Software Foundation, either version 3 of the License, or
(at your option) any later version.

s2Dcd is distributed in the hope that it will be useful,
but WITHOUT ANY WARRANTY; without even the implied warranty of
MERCHANTABILITY or FITNESS FOR A PARTICULAR PURPOSE.  See the
GNU General Public License for more details.

You should have received a copy of the GNU General Public License
along with s2Dcd.  If not, see \textless{}\sphinxurl{https://www.gnu.org/licenses/}\textgreater{}.

\item[{Purpose}] \leavevmode
A module containing some classes and function useful to work with
structured grids.

\item[{This file}] \leavevmode
\sphinxtitleref{grid.py}

\item[{Version}] \leavevmode\begin{description}
\item[{0.3 , 2013\sphinxhyphen{}08\sphinxhyphen{}11 :}] \leavevmode\begin{itemize}
\item {} 
Adapted to the case of grids made of cells or grid made of
points.

\end{itemize}

\item[{0.2 , 2013\sphinxhyphen{}03\sphinxhyphen{}06 :}] \leavevmode\begin{itemize}
\item {} 
Converted to Python3.X with \sphinxtitleref{2to3}

\end{itemize}

\item[{0.1 , 2012\sphinxhyphen{}04\sphinxhyphen{}02 :}] \leavevmode\begin{itemize}
\item {} 
First version.

\end{itemize}

\end{description}

\item[{Authors}] \leavevmode
Alessandro Comunian

\end{description}\end{quote}

\begin{sphinxadmonition}{note}{Note:}
The grids for the moment are always considered as 3D.
\end{sphinxadmonition}
\index{Grid (class in s2Dcd.grid)@\spxentry{Grid}\spxextra{class in s2Dcd.grid}}

\begin{fulllineitems}
\phantomsection\label{\detokenize{appendices:s2Dcd.grid.Grid}}\pysiglinewithargsret{\sphinxbfcode{\sphinxupquote{class }}\sphinxcode{\sphinxupquote{s2Dcd.grid.}}\sphinxbfcode{\sphinxupquote{Grid}}}{\emph{ox=0.0}, \emph{oy=0.0}, \emph{oz=0.0}, \emph{dx=1.0}, \emph{dy=1.0}, \emph{dz=1.0}, \emph{nx=200}, \emph{ny=200}, \emph{nz=1}, \emph{gtype=\textquotesingle{}points\textquotesingle{}}}{}
A simple class that contains the \sphinxstyleemphasis{Origin}, \sphinxstyleemphasis{Delta} and \sphinxstyleemphasis{Size} of a
simulation.  It can also be used as container for some information
contained in a VTK structured grid file.

\begin{sphinxadmonition}{note}{Note:}\begin{itemize}
\item {} 
By default, the size of the grid is considered in term of points.

\end{itemize}
\end{sphinxadmonition}


\sphinxstrong{See also:}


\sphinxcode{\sphinxupquote{vtknumpy}}


\index{cells() (s2Dcd.grid.Grid property)@\spxentry{cells()}\spxextra{s2Dcd.grid.Grid property}}

\begin{fulllineitems}
\phantomsection\label{\detokenize{appendices:s2Dcd.grid.Grid.cells}}\pysigline{\sphinxbfcode{\sphinxupquote{property }}\sphinxbfcode{\sphinxupquote{cells}}}
Number of cells

\end{fulllineitems}

\index{compute\_max() (s2Dcd.grid.Grid method)@\spxentry{compute\_max()}\spxextra{s2Dcd.grid.Grid method}}

\begin{fulllineitems}
\phantomsection\label{\detokenize{appendices:s2Dcd.grid.Grid.compute_max}}\pysiglinewithargsret{\sphinxbfcode{\sphinxupquote{compute\_max}}}{}{}
Compute the max values for \sphinxstyleemphasis{x}, \sphinxstyleemphasis{y} and \sphinxstyleemphasis{z} of the grid.

\end{fulllineitems}

\index{get\_cells() (s2Dcd.grid.Grid method)@\spxentry{get\_cells()}\spxextra{s2Dcd.grid.Grid method}}

\begin{fulllineitems}
\phantomsection\label{\detokenize{appendices:s2Dcd.grid.Grid.get_cells}}\pysiglinewithargsret{\sphinxbfcode{\sphinxupquote{get\_cells}}}{}{}
Update the number of cells for a cells grid

\end{fulllineitems}

\index{get\_center() (s2Dcd.grid.Grid method)@\spxentry{get\_center()}\spxextra{s2Dcd.grid.Grid method}}

\begin{fulllineitems}
\phantomsection\label{\detokenize{appendices:s2Dcd.grid.Grid.get_center}}\pysiglinewithargsret{\sphinxbfcode{\sphinxupquote{get\_center}}}{}{}
Returns the center of the grid

\end{fulllineitems}

\index{get\_lx() (s2Dcd.grid.Grid method)@\spxentry{get\_lx()}\spxextra{s2Dcd.grid.Grid method}}

\begin{fulllineitems}
\phantomsection\label{\detokenize{appendices:s2Dcd.grid.Grid.get_lx}}\pysiglinewithargsret{\sphinxbfcode{\sphinxupquote{get\_lx}}}{}{}
Provide as output a tuple containing the size of a
grid. Useful for the implementation of \sphinxcode{\sphinxupquote{property}}.

\end{fulllineitems}

\index{get\_ly() (s2Dcd.grid.Grid method)@\spxentry{get\_ly()}\spxextra{s2Dcd.grid.Grid method}}

\begin{fulllineitems}
\phantomsection\label{\detokenize{appendices:s2Dcd.grid.Grid.get_ly}}\pysiglinewithargsret{\sphinxbfcode{\sphinxupquote{get\_ly}}}{}{}
Provide as output a tuple containing the size of a
grid. Useful for the implementation of \sphinxcode{\sphinxupquote{property}}.

\end{fulllineitems}

\index{get\_lz() (s2Dcd.grid.Grid method)@\spxentry{get\_lz()}\spxextra{s2Dcd.grid.Grid method}}

\begin{fulllineitems}
\phantomsection\label{\detokenize{appendices:s2Dcd.grid.Grid.get_lz}}\pysiglinewithargsret{\sphinxbfcode{\sphinxupquote{get\_lz}}}{}{}
Provide as output a tuple containing the size of a
grid. Useful for the implementation of \sphinxcode{\sphinxupquote{property}}.

\end{fulllineitems}

\index{get\_points() (s2Dcd.grid.Grid method)@\spxentry{get\_points()}\spxextra{s2Dcd.grid.Grid method}}

\begin{fulllineitems}
\phantomsection\label{\detokenize{appendices:s2Dcd.grid.Grid.get_points}}\pysiglinewithargsret{\sphinxbfcode{\sphinxupquote{get\_points}}}{}{}
Update the number of points for a points grid

\end{fulllineitems}

\index{get\_size() (s2Dcd.grid.Grid method)@\spxentry{get\_size()}\spxextra{s2Dcd.grid.Grid method}}

\begin{fulllineitems}
\phantomsection\label{\detokenize{appendices:s2Dcd.grid.Grid.get_size}}\pysiglinewithargsret{\sphinxbfcode{\sphinxupquote{get\_size}}}{}{}
Compute the total number of points/cells in the grid.

\end{fulllineitems}

\index{lx() (s2Dcd.grid.Grid property)@\spxentry{lx()}\spxextra{s2Dcd.grid.Grid property}}

\begin{fulllineitems}
\phantomsection\label{\detokenize{appendices:s2Dcd.grid.Grid.lx}}\pysigline{\sphinxbfcode{\sphinxupquote{property }}\sphinxbfcode{\sphinxupquote{lx}}}
‘size’ along \sphinxstyleemphasis{x} of the grid.

\end{fulllineitems}

\index{ly() (s2Dcd.grid.Grid property)@\spxentry{ly()}\spxextra{s2Dcd.grid.Grid property}}

\begin{fulllineitems}
\phantomsection\label{\detokenize{appendices:s2Dcd.grid.Grid.ly}}\pysigline{\sphinxbfcode{\sphinxupquote{property }}\sphinxbfcode{\sphinxupquote{ly}}}
‘size’ along \sphinxstyleemphasis{y} of the grid.

\end{fulllineitems}

\index{lz() (s2Dcd.grid.Grid property)@\spxentry{lz()}\spxextra{s2Dcd.grid.Grid property}}

\begin{fulllineitems}
\phantomsection\label{\detokenize{appendices:s2Dcd.grid.Grid.lz}}\pysigline{\sphinxbfcode{\sphinxupquote{property }}\sphinxbfcode{\sphinxupquote{lz}}}
‘size’ along \sphinxstyleemphasis{z} of the grid.

\end{fulllineitems}

\index{origin() (s2Dcd.grid.Grid method)@\spxentry{origin()}\spxextra{s2Dcd.grid.Grid method}}

\begin{fulllineitems}
\phantomsection\label{\detokenize{appendices:s2Dcd.grid.Grid.origin}}\pysiglinewithargsret{\sphinxbfcode{\sphinxupquote{origin}}}{}{}
To print out the origin of the grid as a tuple
\begin{description}
\item[{Parameters:}] \leavevmode
self : an instance of the Grid class

\item[{Returns:}] \leavevmode
A tuple containing the origin defined in the grid.

\end{description}

\end{fulllineitems}

\index{points() (s2Dcd.grid.Grid property)@\spxentry{points()}\spxextra{s2Dcd.grid.Grid property}}

\begin{fulllineitems}
\phantomsection\label{\detokenize{appendices:s2Dcd.grid.Grid.points}}\pysigline{\sphinxbfcode{\sphinxupquote{property }}\sphinxbfcode{\sphinxupquote{points}}}
Number of points

\end{fulllineitems}

\index{print4gslib() (s2Dcd.grid.Grid method)@\spxentry{print4gslib()}\spxextra{s2Dcd.grid.Grid method}}

\begin{fulllineitems}
\phantomsection\label{\detokenize{appendices:s2Dcd.grid.Grid.print4gslib}}\pysiglinewithargsret{\sphinxbfcode{\sphinxupquote{print4gslib}}}{}{}
A function to print out the grid information in the header of
a GSLIB file.

\end{fulllineitems}

\index{print\_intervals() (s2Dcd.grid.Grid method)@\spxentry{print\_intervals()}\spxextra{s2Dcd.grid.Grid method}}

\begin{fulllineitems}
\phantomsection\label{\detokenize{appendices:s2Dcd.grid.Grid.print_intervals}}\pysiglinewithargsret{\sphinxbfcode{\sphinxupquote{print\_intervals}}}{\emph{axis=\textquotesingle{}xyz\textquotesingle{}}}{}
Print the intervals that constitute the simulation domain in a
format like:

\begin{sphinxVerbatim}[commandchars=\\\{\}]
\PYG{p}{[} \PYG{n}{ox}\PYG{p}{,} \PYG{n}{ox}\PYG{o}{+}\PYG{n}{nx}\PYG{o}{*}\PYG{n}{dx}\PYG{p}{]} \PYG{p}{[} \PYG{n}{oy}\PYG{p}{,} \PYG{n}{oy}\PYG{o}{+}\PYG{n}{ny}\PYG{o}{*}\PYG{n}{dy}\PYG{p}{]} \PYG{p}{[} \PYG{n}{oz}\PYG{p}{,} \PYG{n}{oz}\PYG{o}{+}\PYG{n}{nz}\PYG{o}{*}\PYG{n}{dz}\PYG{p}{]}
\end{sphinxVerbatim}

where \sphinxstyleemphasis{ox} is the origin, \sphinxstyleemphasis{nx} is the number of points and \sphinxstyleemphasis{dx}
is the delta between points (\sphinxstyleemphasis{idem} for \sphinxstyleemphasis{y} and \sphinxstyleemphasis{z}).
\begin{description}
\item[{Parameters:}] \leavevmode\begin{description}
\item[{axis: string containing {[}‘x’,’y’,’z’{]}, optional}] \leavevmode
If the default value “xyz” is used, then all the intevals
are printed.

\end{description}

\end{description}

\end{fulllineitems}

\index{shape() (s2Dcd.grid.Grid method)@\spxentry{shape()}\spxextra{s2Dcd.grid.Grid method}}

\begin{fulllineitems}
\phantomsection\label{\detokenize{appendices:s2Dcd.grid.Grid.shape}}\pysiglinewithargsret{\sphinxbfcode{\sphinxupquote{shape}}}{}{}
To print out the shape of the grid as a tuple

Parameters:
\begin{quote}

self:
\begin{quote}

an instance of the Grid class
\end{quote}

gtype: string in (‘points’, ‘cells’)
\begin{quote}

String to decide to print the shape in terms of points
or in terms of cells.
\end{quote}
\end{quote}
\begin{description}
\item[{Returns:}] \leavevmode
A tuple containing the origin defined in the grid.

\end{description}

\end{fulllineitems}

\index{spacing() (s2Dcd.grid.Grid method)@\spxentry{spacing()}\spxextra{s2Dcd.grid.Grid method}}

\begin{fulllineitems}
\phantomsection\label{\detokenize{appendices:s2Dcd.grid.Grid.spacing}}\pysiglinewithargsret{\sphinxbfcode{\sphinxupquote{spacing}}}{}{}
To print out the spacing of the grid as a tuple
\begin{description}
\item[{Parameters:}] \leavevmode
self : an instance of the Grid class

\item[{Returns:}] \leavevmode
A tuple  containing the spacing defined in the grid.

\end{description}

\end{fulllineitems}


\end{fulllineitems}

\index{sg\_info() (in module s2Dcd.grid)@\spxentry{sg\_info()}\spxextra{in module s2Dcd.grid}}

\begin{fulllineitems}
\phantomsection\label{\detokenize{appendices:s2Dcd.grid.sg_info}}\pysiglinewithargsret{\sphinxcode{\sphinxupquote{s2Dcd.grid.}}\sphinxbfcode{\sphinxupquote{sg\_info}}}{\emph{data}}{}
Collect some information about a structured grid
contained in a numpy array

\end{fulllineitems}



\section{The \sphinxstyleliteralintitle{\sphinxupquote{s2Dcd.gslibnumpy}} module}
\label{\detokenize{appendices:module-s2Dcd.gslibnumpy}}\label{\detokenize{appendices:the-s2dcd-gslibnumpy-module}}\index{s2Dcd.gslibnumpy (module)@\spxentry{s2Dcd.gslibnumpy}\spxextra{module}}\begin{quote}\begin{description}
\item[{license}] \leavevmode
This file is part of s2Dcd.

s2Dcd is free software: you can redistribute it and/or modify
it under the terms of the GNU General Public License as published by
the Free Software Foundation, either version 3 of the License, or
(at your option) any later version.

s2Dcd is distributed in the hope that it will be useful,
but WITHOUT ANY WARRANTY; without even the implied warranty of
MERCHANTABILITY or FITNESS FOR A PARTICULAR PURPOSE.  See the
GNU General Public License for more details.

You should have received a copy of the GNU General Public License
along with s2Dcd.  If not, see \textless{}\sphinxurl{https://www.gnu.org/licenses/}\textgreater{}.

\item[{Purpose}] \leavevmode
A module with some utilities to convert from numpy to gslib and
vice\sphinxhyphen{}versa, and to support some GSLIB based software.

For more details about the GSLIB software, see \sphinxcite{appendices:deutsch1988} .

\item[{This file}] \leavevmode
\sphinxcode{\sphinxupquote{gslibnumpy.py}}

\item[{Version}] \leavevmode\begin{description}
\item[{0.9 , 2020\sphinxhyphen{}01\sphinxhyphen{}20 :}] \leavevmode\begin{itemize}
\item {} 
Adapted to the inclusion in the s2Dcd package on Github.

\end{itemize}

\item[{0.8 , 2015\sphinxhyphen{}12\sphinxhyphen{}02 :}] \leavevmode\begin{itemize}
\item {} 
Added \sphinxtitleref{gslib\_slice}.

\item {} 
General clean\sphinxhyphen{}up

\item {} 
Improved numpy2gslib to allow printing many variables

\item {} 
Improved the pandas usage in numpy2gslib

\end{itemize}

\item[{0.7 , 2014\sphinxhyphen{}11\sphinxhyphen{}13 :}] \leavevmode\begin{itemize}
\item {} 
Introduced dependencies from “pandas” to load txt faster

\end{itemize}

\item[{0.6 , 2013\sphinxhyphen{}08\sphinxhyphen{}26 :}] \leavevmode\begin{itemize}
\item {} 
Improved how errors are handled in \sphinxtitleref{gslib2numpy} and the
extension available.

\end{itemize}

\item[{0.5 , 2013\sphinxhyphen{}07\sphinxhyphen{}07 :}] \leavevmode\begin{itemize}
\item {} 
Improved the function \sphinxtitleref{gslib\_points2numpy}.

\end{itemize}

\item[{0.4 , 2013\sphinxhyphen{}03\sphinxhyphen{}06 :}] \leavevmode\begin{itemize}
\item {} 
Converted to Python3 using \sphinxtitleref{2to3}.

\item {} 
Improved the flexibility of \sphinxtitleref{gslib2numpy}.

\item {} 
Remover the dependence from the module \sphinxtitleref{gslibnumpy\_f}.

\end{itemize}

\item[{0.3 , 2012\sphinxhyphen{}11\sphinxhyphen{}19 :}] \leavevmode\begin{itemize}
\item {} 
Last version before the conversion to Python3.

\item {} 
Module \sphinxtitleref{gslib2numpy} improved to be able to read also files
containing multiple variables.

\item {} 
merged the function \sphinxtitleref{dat2numpy} into \sphinxtitleref{gslib2numpy}.

\end{itemize}

\item[{0.2 , 2012\sphinxhyphen{}11\sphinxhyphen{}18 :}] \leavevmode\begin{itemize}
\item {} 
Some improvement using the suggestions of Spyder.

\item {} 
Deleted the option for the “dtype” in function
“gslib\_points2numpy”.

\item {} 
included a new function to load GSLIB files coming from Isatis
(dat2numpy, for files \sphinxcode{\sphinxupquote{.dat}})

\end{itemize}

\item[{0.1 , 2012\sphinxhyphen{}04\sphinxhyphen{}27 :}] \leavevmode\begin{itemize}
\item {} 
Implemented in FORTRAN90 the subroutines \sphinxtitleref{numpy2gslib}.

\end{itemize}

\item[{0.0 , 2012\sphinxhyphen{}03\sphinxhyphen{}07:}] \leavevmode\begin{itemize}
\item {} 
First version.

\end{itemize}

\end{description}

\item[{Authors}] \leavevmode
Alessandro Comunian

\end{description}\end{quote}


\subsection{References}
\label{\detokenize{appendices:references}}

\subsection{Functions and classes}
\label{\detokenize{appendices:functions-and-classes}}\index{VarioStruct (class in s2Dcd.gslibnumpy)@\spxentry{VarioStruct}\spxextra{class in s2Dcd.gslibnumpy}}

\begin{fulllineitems}
\phantomsection\label{\detokenize{appendices:s2Dcd.gslibnumpy.VarioStruct}}\pysiglinewithargsret{\sphinxbfcode{\sphinxupquote{class }}\sphinxcode{\sphinxupquote{s2Dcd.gslibnumpy.}}\sphinxbfcode{\sphinxupquote{VarioStruct}}}{\emph{stype=1, c\_par=1.0, angles={[}0.0, 0.0, 0.0{]}, ranges={[}100.0, 100.0, 100.0{]}}}{}
A class to contain info about one structure of a variogram
\index{print4csv() (s2Dcd.gslibnumpy.VarioStruct method)@\spxentry{print4csv()}\spxextra{s2Dcd.gslibnumpy.VarioStruct method}}

\begin{fulllineitems}
\phantomsection\label{\detokenize{appendices:s2Dcd.gslibnumpy.VarioStruct.print4csv}}\pysiglinewithargsret{\sphinxbfcode{\sphinxupquote{print4csv}}}{}{}
Print a string with the information contained in the class
useful in a CSV file.

\end{fulllineitems}

\index{print4par() (s2Dcd.gslibnumpy.VarioStruct method)@\spxentry{print4par()}\spxextra{s2Dcd.gslibnumpy.VarioStruct method}}

\begin{fulllineitems}
\phantomsection\label{\detokenize{appendices:s2Dcd.gslibnumpy.VarioStruct.print4par}}\pysiglinewithargsret{\sphinxbfcode{\sphinxupquote{print4par}}}{}{}
Print a string with the information contained in the class
useful in a parameter file.

\end{fulllineitems}


\end{fulllineitems}

\index{Variogram (class in s2Dcd.gslibnumpy)@\spxentry{Variogram}\spxextra{class in s2Dcd.gslibnumpy}}

\begin{fulllineitems}
\phantomsection\label{\detokenize{appendices:s2Dcd.gslibnumpy.Variogram}}\pysiglinewithargsret{\sphinxbfcode{\sphinxupquote{class }}\sphinxcode{\sphinxupquote{s2Dcd.gslibnumpy.}}\sphinxbfcode{\sphinxupquote{Variogram}}}{\emph{str\_nb=1}, \emph{nugget=0.0}, \emph{cat\_thr=0}}{}
A class to contain all the parameters related to a GSLIB variogram
model.
\index{plot() (s2Dcd.gslibnumpy.Variogram method)@\spxentry{plot()}\spxextra{s2Dcd.gslibnumpy.Variogram method}}

\begin{fulllineitems}
\phantomsection\label{\detokenize{appendices:s2Dcd.gslibnumpy.Variogram.plot}}\pysiglinewithargsret{\sphinxbfcode{\sphinxupquote{plot}}}{\emph{file\_name}, \emph{h\_max=200.0}}{}
Plot the variogram model using matplotlib
\begin{description}
\item[{Parameters:}] \leavevmode\begin{description}
\item[{h\_max: float}] \leavevmode
Maximum value of the lag

\end{description}

\end{description}

\end{fulllineitems}

\index{plot\_3dir() (s2Dcd.gslibnumpy.Variogram method)@\spxentry{plot\_3dir()}\spxextra{s2Dcd.gslibnumpy.Variogram method}}

\begin{fulllineitems}
\phantomsection\label{\detokenize{appendices:s2Dcd.gslibnumpy.Variogram.plot_3dir}}\pysiglinewithargsret{\sphinxbfcode{\sphinxupquote{plot\_3dir}}}{\emph{file\_name}, \emph{h\_max=None}, \emph{suptitle=None}}{}
Plot the variogram model using matplotlib along all the three
direction for all the available ranges.
\begin{description}
\item[{Parameters:}] \leavevmode\begin{description}
\item[{h\_max: float}] \leavevmode
Maximum value of the lag

\item[{suptitle: string}] \leavevmode
A string for the super\sphinxhyphen{}title.

\end{description}

\end{description}

\begin{sphinxadmonition}{note}{Note:}
Use by default the range of the 1st structure to define the 
upper limit of the \(x\) axis.
\end{sphinxadmonition}

\end{fulllineitems}

\index{print4csv() (s2Dcd.gslibnumpy.Variogram method)@\spxentry{print4csv()}\spxextra{s2Dcd.gslibnumpy.Variogram method}}

\begin{fulllineitems}
\phantomsection\label{\detokenize{appendices:s2Dcd.gslibnumpy.Variogram.print4csv}}\pysiglinewithargsret{\sphinxbfcode{\sphinxupquote{print4csv}}}{}{}
Print the info about a variogram for a CSV file

\end{fulllineitems}

\index{print4par() (s2Dcd.gslibnumpy.Variogram method)@\spxentry{print4par()}\spxextra{s2Dcd.gslibnumpy.Variogram method}}

\begin{fulllineitems}
\phantomsection\label{\detokenize{appendices:s2Dcd.gslibnumpy.Variogram.print4par}}\pysiglinewithargsret{\sphinxbfcode{\sphinxupquote{print4par}}}{}{}
Print the info about a variogram for a parameter file

\end{fulllineitems}


\end{fulllineitems}

\index{align\_parfile() (in module s2Dcd.gslibnumpy)@\spxentry{align\_parfile()}\spxextra{in module s2Dcd.gslibnumpy}}

\begin{fulllineitems}
\phantomsection\label{\detokenize{appendices:s2Dcd.gslibnumpy.align_parfile}}\pysiglinewithargsret{\sphinxcode{\sphinxupquote{s2Dcd.gslibnumpy.}}\sphinxbfcode{\sphinxupquote{align\_parfile}}}{\emph{par\_file}}{}
Read a string as a parfile, detect where the character “” is and
align everything.
\begin{description}
\item[{Parameters:}] \leavevmode\begin{description}
\item[{par\_file: string}] \leavevmode
A string containing the parameter file to be indented.

\end{description}

\item[{Returns:}] \leavevmode
A string containing the parameter file indented.

\end{description}

\end{fulllineitems}

\index{exponential() (in module s2Dcd.gslibnumpy)@\spxentry{exponential()}\spxextra{in module s2Dcd.gslibnumpy}}

\begin{fulllineitems}
\phantomsection\label{\detokenize{appendices:s2Dcd.gslibnumpy.exponential}}\pysiglinewithargsret{\sphinxcode{\sphinxupquote{s2Dcd.gslibnumpy.}}\sphinxbfcode{\sphinxupquote{exponential}}}{\emph{h}, \emph{c=1.0}, \emph{a=100.0}}{}
Definition of the model of a Exponential variogram.
\begin{description}
\item[{Parameters:}] \leavevmode\begin{description}
\item[{h: float}] \leavevmode
The values of lag \(h\) where to plot the variogram

\item[{c: float}] \leavevmode
The sill value \(c\)

\item[{a: float}] \leavevmode
The actual range \(a\)

\end{description}

\item[{Returns:}] \leavevmode
The values of the variogram \(\gamma(h)\)

\end{description}

\end{fulllineitems}

\index{gaussian() (in module s2Dcd.gslibnumpy)@\spxentry{gaussian()}\spxextra{in module s2Dcd.gslibnumpy}}

\begin{fulllineitems}
\phantomsection\label{\detokenize{appendices:s2Dcd.gslibnumpy.gaussian}}\pysiglinewithargsret{\sphinxcode{\sphinxupquote{s2Dcd.gslibnumpy.}}\sphinxbfcode{\sphinxupquote{gaussian}}}{\emph{h}, \emph{c=1.0}, \emph{a=100.0}}{}
Definition of the model of a Gaussian variogram.
\begin{description}
\item[{Parameters:}] \leavevmode\begin{description}
\item[{h: float}] \leavevmode
The values of lag \(h\) where to plot the variogram

\item[{c: float}] \leavevmode
The sill value \(c\)

\item[{a: float}] \leavevmode
The actual range \(a\)

\end{description}

\item[{Returns:}] \leavevmode
The values of the variogram \(\gamma(h)\)

\end{description}

\end{fulllineitems}

\index{gslib2numpy() (in module s2Dcd.gslibnumpy)@\spxentry{gslib2numpy()}\spxextra{in module s2Dcd.gslibnumpy}}

\begin{fulllineitems}
\phantomsection\label{\detokenize{appendices:s2Dcd.gslibnumpy.gslib2numpy}}\pysiglinewithargsret{\sphinxcode{\sphinxupquote{s2Dcd.gslibnumpy.}}\sphinxbfcode{\sphinxupquote{gslib2numpy}}}{\emph{file\_name}, \emph{verbose=True}, \emph{formats=None}, \emph{names=None}, \emph{source=None}, \emph{grd=None}}{}
Convert a GSLIB ASCII grid into a numpy array.

Can convert only a subclass of GSLIB ASCII files.
See note for more details.
\begin{description}
\item[{Parameters:}] \leavevmode\begin{description}
\item[{file\_name: string}] \leavevmode
Name of the GSLIB input file.

\item[{verbose: bool, optional}] \leavevmode
A flag to print reading information or not.

\item[{formats: list of characters, optional (default=None)}] \leavevmode
The definition of the input format, \sphinxcode{\sphinxupquote{numpy.genfromtxt()}}
like.
If nothing is provided, float values are considered for each
variable.

\item[{names: tuple, optional (default=None)}] \leavevmode
The definition of the names of the variables. If not
provided, the names of the variables contained in the
GSLIB file are used.

\item[{source: string in {[}‘impala’,’isatis’, ‘fluvsim’{]},}] \leavevmode
optional (default=None)
The header changes a little if the file is to be used with
“impala”, or it is an output from “isatis”.  The functions
tries to extrapolate from the file extension if the file
comes from “impala” (.gslib) or from “Isatis” (.dat).  If
this argument is provided, then the format of the header
is forced according to this (see the note for more
details).  The output files from FLUVSIM (.out) have a
header that contains only the dimensions of the grid, and
therefore they fall as a simple case of “implala” files.

\end{description}

\item[{Returns:}] \leavevmode\begin{description}
\item[{out\_dict:}] \leavevmode
A numpy dictionary of arrays with the data contained in
the GSLIB file.

\item[{in\_grd:}] \leavevmode
A “grid.Grid” object, containing the grid definition (if
provided).

\end{description}

\end{description}

\begin{sphinxadmonition}{note}{Note:}
Limited to a subsclass of GSLIB ASCII files.
The files can have an “impala” (of FLUVSIM) format, like:

\begin{sphinxVerbatim}[commandchars=\\\{\}]
\PYG{n}{nx} \PYG{n}{ny} \PYG{n}{nz} \PYG{p}{[} \PYG{n}{dx} \PYG{n}{dy} \PYG{n}{dz} \PYG{p}{[} \PYG{n}{ox} \PYG{n}{oy} \PYG{n}{oz}\PYG{p}{]} \PYG{p}{]}
\PYG{n}{nb\PYGZus{}var}
\PYG{n}{var\PYGZus{}name}\PYG{p}{[}\PYG{l+m+mi}{0}\PYG{p}{]}
\PYG{n}{var\PYGZus{}name}\PYG{p}{[}\PYG{l+m+mi}{1}\PYG{p}{]}
\PYG{o}{.}\PYG{o}{.}\PYG{o}{.}
\PYG{n}{var\PYGZus{}name}\PYG{p}{[}\PYG{n}{nb\PYGZus{}var}\PYG{o}{\PYGZhy{}}\PYG{l+m+mi}{1}\PYG{p}{]}
\PYG{n}{var0\PYGZus{}value}\PYG{p}{[}\PYG{l+m+mi}{0}\PYG{p}{]} \PYG{n}{var1\PYGZus{}value}\PYG{p}{[}\PYG{l+m+mi}{0}\PYG{p}{]} \PYG{o}{.}\PYG{o}{.}\PYG{o}{.}
\PYG{n}{var0\PYGZus{}value}\PYG{p}{[}\PYG{l+m+mi}{1}\PYG{p}{]} \PYG{n}{var1\PYGZus{}value}\PYG{p}{[}\PYG{l+m+mi}{1}\PYG{p}{]} \PYG{o}{.}\PYG{o}{.}\PYG{o}{.}
\PYG{o}{.}\PYG{o}{.}\PYG{o}{.}           \PYG{o}{.}\PYG{o}{.}\PYG{o}{.}           \PYG{o}{.}\PYG{o}{.}\PYG{o}{.}
\end{sphinxVerbatim}

Or, in case the software is “Isatis”, then the header will be
like:

\begin{sphinxVerbatim}[commandchars=\\\{\}]
\PYG{n}{Description} \PYG{n}{line}
\PYG{n}{nb\PYGZus{}var} \PYG{n}{nx} \PYG{n}{ny} \PYG{n}{nz} \PYG{n}{ox} \PYG{n}{oy} \PYG{n}{oz} \PYG{n}{dx} \PYG{n}{dy} \PYG{n}{dz}
\PYG{o}{.}\PYG{o}{.}\PYG{o}{.}
\end{sphinxVerbatim}
\end{sphinxadmonition}


\sphinxstrong{See also:}


\sphinxcode{\sphinxupquote{numpy.genfromtxt()}}



\end{fulllineitems}

\index{gslib2numpy\_onevar() (in module s2Dcd.gslibnumpy)@\spxentry{gslib2numpy\_onevar()}\spxextra{in module s2Dcd.gslibnumpy}}

\begin{fulllineitems}
\phantomsection\label{\detokenize{appendices:s2Dcd.gslibnumpy.gslib2numpy_onevar}}\pysiglinewithargsret{\sphinxcode{\sphinxupquote{s2Dcd.gslibnumpy.}}\sphinxbfcode{\sphinxupquote{gslib2numpy\_onevar}}}{\emph{file\_name}, \emph{verbose=True}, \emph{formats=None}, \emph{names=None}, \emph{source=None}}{}
Convert a GSLIB ASCII grid into a numpy array.

This is a simplified version of \sphinxcode{\sphinxupquote{gslib2numpy}} that can be used
when you only have one variable and you don’t care about the grid
information contained in the file.

Can convert only a subclass of GSLIB ASCII files.
See note for more details.
\begin{description}
\item[{Parameters:}] \leavevmode\begin{description}
\item[{file\_name: string}] \leavevmode
Name of the GSLIB input file.

\item[{verbose: bool, optional}] \leavevmode
A flag to print reading information or not.

\item[{formats: list of characters, optional (default=None)}] \leavevmode
The definition of the input format, \sphinxcode{\sphinxupquote{numpy.genfromtxt()}}
like.
If nothing is provided, float values are considered for each
variable.

\item[{names: tuple, optional (default=None)}] \leavevmode
The definition of the names of the variables. If not provided,
the names of the variables contained in the GSLIB file are used.

\item[{source: string in {[}‘impala’,’isatis’{]}, optional (default=None)}] \leavevmode
The header changes a little if the file is to be used with 
“impala”, or it is an output from “isatis”.
The functions tries to extrapolate from the file extension if
the file comes from “impala” (\sphinxstyleemphasis{.gslib) or from “Isatis” (}.dat).
If this argument is provided, then the format of the header is
forced according to this (see the note for more details).

\end{description}

\item[{Returns:}] \leavevmode\begin{description}
\item[{out\_dict:}] \leavevmode
A numpy dictionary of arrays with the data contained in
the GSLIB file.  NON NON …. ONLY A VARIABLE! UPDATE THIS
DOCUMENTATION IF IT WORKS.

\item[{in\_grd:}] \leavevmode
A “grid.Grid” object, containing the grid definition (if
provided).

\end{description}

\end{description}

\begin{sphinxadmonition}{note}{Note:}
Limited to a subsclass of GSLIB ASCII files.
The files can have an “impala” format, like:

\begin{sphinxVerbatim}[commandchars=\\\{\}]
\PYG{n}{nx} \PYG{n}{ny} \PYG{n}{nz} \PYG{p}{[} \PYG{n}{dx} \PYG{n}{dy} \PYG{n}{dz} \PYG{p}{[} \PYG{n}{ox} \PYG{n}{oy} \PYG{n}{oz}\PYG{p}{]} \PYG{p}{]}
\PYG{n}{nb\PYGZus{}var}
\PYG{n}{var\PYGZus{}name}\PYG{p}{[}\PYG{l+m+mi}{0}\PYG{p}{]}
\PYG{n}{var\PYGZus{}name}\PYG{p}{[}\PYG{l+m+mi}{1}\PYG{p}{]}
\PYG{o}{.}\PYG{o}{.}\PYG{o}{.}
\PYG{n}{var\PYGZus{}name}\PYG{p}{[}\PYG{n}{nb\PYGZus{}var}\PYG{o}{\PYGZhy{}}\PYG{l+m+mi}{1}\PYG{p}{]}
\PYG{n}{var0\PYGZus{}value}\PYG{p}{[}\PYG{l+m+mi}{0}\PYG{p}{]} \PYG{n}{var1\PYGZus{}value}\PYG{p}{[}\PYG{l+m+mi}{0}\PYG{p}{]} \PYG{o}{.}\PYG{o}{.}\PYG{o}{.}
\PYG{n}{var0\PYGZus{}value}\PYG{p}{[}\PYG{l+m+mi}{1}\PYG{p}{]} \PYG{n}{var1\PYGZus{}value}\PYG{p}{[}\PYG{l+m+mi}{1}\PYG{p}{]} \PYG{o}{.}\PYG{o}{.}\PYG{o}{.}
\PYG{o}{.}\PYG{o}{.}\PYG{o}{.}           \PYG{o}{.}\PYG{o}{.}\PYG{o}{.}           \PYG{o}{.}\PYG{o}{.}\PYG{o}{.}
\end{sphinxVerbatim}

Or, in case the software is “Isatis”, then the header will be like:

\begin{sphinxVerbatim}[commandchars=\\\{\}]
\PYG{n}{Description} \PYG{n}{line}
\PYG{n}{nb\PYGZus{}var} \PYG{n}{nx} \PYG{n}{ny} \PYG{n}{nz} \PYG{n}{ox} \PYG{n}{oy} \PYG{n}{oz} \PYG{n}{dx} \PYG{n}{dy} \PYG{n}{dz}
\PYG{o}{.}\PYG{o}{.}\PYG{o}{.}
\end{sphinxVerbatim}
\end{sphinxadmonition}


\sphinxstrong{See also:}


\sphinxcode{\sphinxupquote{numpy.genfromtxt()}}



\end{fulllineitems}

\index{gslib\_points2numpy() (in module s2Dcd.gslibnumpy)@\spxentry{gslib\_points2numpy()}\spxextra{in module s2Dcd.gslibnumpy}}

\begin{fulllineitems}
\phantomsection\label{\detokenize{appendices:s2Dcd.gslibnumpy.gslib_points2numpy}}\pysiglinewithargsret{\sphinxcode{\sphinxupquote{s2Dcd.gslibnumpy.}}\sphinxbfcode{\sphinxupquote{gslib\_points2numpy}}}{\emph{file\_name}, \emph{verbose=True}, \emph{formats=None}}{}
Convert to a GSLIB ASCII point data file into a numpy array.
\begin{description}
\item[{Parameters:}] \leavevmode\begin{description}
\item[{file\_name}] \leavevmode{[}string{]}
Name of the GSLIB input file.

\item[{verbose}] \leavevmode{[}bool, optional{]}
A flag to print out information or not.

\item[{formats: list of characters, optional (default=None)}] \leavevmode
The definition of the input format,
\sphinxcode{\sphinxupquote{numpy.genfromtxt()}} like.  If nothing is provided,
then \sphinxtitleref{genfromtxt} tries to understand the format
automatically.

\end{description}

\item[{Returns:}] \leavevmode
An \sphinxtitleref{ndarray} containing a key for each variable contained in
the GSLIB file and a string containing the header of the file.

\item[{Example:}] \leavevmode
Read the content of the file \sphinxtitleref{test.gslib}:

\begin{sphinxVerbatim}[commandchars=\\\{\}]
\PYG{g+gp}{\PYGZgt{}\PYGZgt{}\PYGZgt{} }\PYG{n}{data} \PYG{o}{=} \PYG{n}{gslib\PYGZus{}points2numpy}\PYG{p}{(}\PYG{l+s+s2}{\PYGZdq{}}\PYG{l+s+s2}{test.gslib}\PYG{l+s+s2}{\PYGZdq{}}\PYG{p}{)}
\PYG{g+gp}{\PYGZgt{}\PYGZgt{}\PYGZgt{} }\PYG{n}{x} \PYG{o}{=} \PYG{n}{data}\PYG{p}{[}\PYG{l+s+s1}{\PYGZsq{}}\PYG{l+s+s1}{x}\PYG{l+s+s1}{\PYGZsq{}}\PYG{p}{]}
\end{sphinxVerbatim}

\end{description}

\end{fulllineitems}

\index{gslib\_slice() (in module s2Dcd.gslibnumpy)@\spxentry{gslib\_slice()}\spxextra{in module s2Dcd.gslibnumpy}}

\begin{fulllineitems}
\phantomsection\label{\detokenize{appendices:s2Dcd.gslibnumpy.gslib_slice}}\pysiglinewithargsret{\sphinxcode{\sphinxupquote{s2Dcd.gslibnumpy.}}\sphinxbfcode{\sphinxupquote{gslib\_slice}}}{\emph{file\_in}, \emph{axis}, \emph{level=None}}{}
Cut a 2D slice within a 3D \sphinxtitleref{GSLIB} file perpendicular to a given
\sphinxstyleemphasis{axis} at a given \sphinxstyleemphasis{level}. The \sphinxtitleref{GSLIB} file can contain more than
one variable.
\begin{description}
\item[{Parameters:}] \leavevmode\begin{description}
\item[{file\_in}] \leavevmode{[}string,{]}
The name of the input file.

\item[{axis}] \leavevmode{[}char, in {[}‘x’, ‘y’, ‘z’{]}{]}
The axis coordinate which is used to cut the slice.

\item[{level}] \leavevmode{[}int, optional (default=None){]}
The index (\sphinxstylestrong{not} the true coordinate) along the
coordinate where \sphinxtitleref{axis} is defined where to cut the slice.
If \sphinxstyleemphasis{None} then an index in the middle of the input
domain is selected.

\end{description}

\item[{Returns:}] \leavevmode
A file with the same name as \sphinxtitleref{file\_in} and a suffix
\sphinxtitleref{\sphinxhyphen{}\textless{}axis\textgreater{}\textless{}level\textgreater{}.gslib} is created, and 0 if successful.
If the return value is \textless{} 0 then there was some problem…

\end{description}

\begin{sphinxadmonition}{warning}{Warning:}
If in the target directory there is a file with the same name as the
file created as return value this is overwritten.
\end{sphinxadmonition}

\end{fulllineitems}

\index{numpy2dat() (in module s2Dcd.gslibnumpy)@\spxentry{numpy2dat()}\spxextra{in module s2Dcd.gslibnumpy}}

\begin{fulllineitems}
\phantomsection\label{\detokenize{appendices:s2Dcd.gslibnumpy.numpy2dat}}\pysiglinewithargsret{\sphinxcode{\sphinxupquote{s2Dcd.gslibnumpy.}}\sphinxbfcode{\sphinxupquote{numpy2dat}}}{\emph{file\_name}, \emph{data}, \emph{grd=None}, \emph{varname=None}}{}
Convert a numpy array into a GSLIB ASCII file in the \sphinxstyleemphasis{Isatis} format.
\begin{description}
\item[{Parameters:}] \leavevmode\begin{description}
\item[{file\_name}] \leavevmode{[}string{]}
The name of the GSLIB file where to save the data.

\item[{data}] \leavevmode{[}numpy array{]}
The numpy array to be saved as GSLIB file.

\item[{grd}] \leavevmode{[}grid.Grid object, optional{]}
The definition of the grid for the dataset.

\item[{varname}] \leavevmode{[}string, optional{]}
The name of the variable to be stored in the GSLIB file.

\end{description}

\end{description}

\begin{sphinxadmonition}{note}{Note:}\begin{itemize}
\item {} 
Works only with 1D, 2D or 3D numpy arrays.

\item {} 
Only one variable per file.

\item {} 
In some cases the input array must be flatten.

\end{itemize}
\end{sphinxadmonition}

\end{fulllineitems}

\index{numpy2gslib() (in module s2Dcd.gslibnumpy)@\spxentry{numpy2gslib()}\spxextra{in module s2Dcd.gslibnumpy}}

\begin{fulllineitems}
\phantomsection\label{\detokenize{appendices:s2Dcd.gslibnumpy.numpy2gslib}}\pysiglinewithargsret{\sphinxcode{\sphinxupquote{s2Dcd.gslibnumpy.}}\sphinxbfcode{\sphinxupquote{numpy2gslib}}}{\emph{data}, \emph{file\_name}, \emph{var\_names=(\textquotesingle{}data\textquotesingle{}}, \emph{)}, \emph{grd=None}, \emph{float\_format=None}, \emph{verbose=False}}{}
Convert a tuple of numpy arrays into a GSLIB ASCII file.
\begin{description}
\item[{Parameters:}] \leavevmode\begin{description}
\item[{data}] \leavevmode{[}tuple of numpy arrays{]}
The numpy arrays to be saved as GSLIB file.

\item[{file\_name}] \leavevmode{[}string (optional){]}
The name of the GSLIB file where to save the data.

\item[{varname}] \leavevmode{[}tuple of string, optional{]}
The names of the variable to be stored in the GSLIB file.
(one for each numpy array in data).

\item[{grd}] \leavevmode{[}grid object, optional{]}
If available, get the dimension of the grid from a grid object.

\item[{float\_format: string (optional, default None)}] \leavevmode
This is useful to pass to pandas \sphinxtitleref{to\_csv} the float format required.

\item[{verbose}] \leavevmode{[}boolean flag, optional{]}
Print or not some info useful for debugging…

\end{description}

\end{description}

\begin{sphinxadmonition}{note}{Note:}\begin{itemize}
\item {} 
Works only with 1D, 2D or 3D numpy arrays.

\item {} 
When more than one variable is provided, the order provided
in “var\_names” is respected when printing the columns.

\item {} 
\sphinxtitleref{data} should be \sphinxstylestrong{always} provided as a tuple, even when
it contains one array only, otherwise it may result in a corrupted
file.

\end{itemize}
\end{sphinxadmonition}

\end{fulllineitems}

\index{numpy2gslib\_points() (in module s2Dcd.gslibnumpy)@\spxentry{numpy2gslib\_points()}\spxextra{in module s2Dcd.gslibnumpy}}

\begin{fulllineitems}
\phantomsection\label{\detokenize{appendices:s2Dcd.gslibnumpy.numpy2gslib_points}}\pysiglinewithargsret{\sphinxcode{\sphinxupquote{s2Dcd.gslibnumpy.}}\sphinxbfcode{\sphinxupquote{numpy2gslib\_points}}}{\emph{xyz\_data}, \emph{file\_name}, \emph{varname=\textquotesingle{}data\textquotesingle{}}}{}
Convert a numpy array into a GSLIB points ASCII file.
\begin{description}
\item[{Parameters:}] \leavevmode\begin{description}
\item[{xyz\_data}] \leavevmode{[}tuple of numpy arrays{]}
The tuple is made of four numpy arrays of the same lenght:
the coordinates x, y, z and the point data \sphinxstyleemphasis{data}.
Depending on the size of the tuple, the points are considered
1D, 2D or 3D.

\item[{file\_name}] \leavevmode{[}string{]}
The name of the GSLIB file where to save the data.

\item[{varname}] \leavevmode{[}string, optional{]}
The name of the variable.

\end{description}

\item[{Returns:}] \leavevmode
A gslib point file containing the coordinates of the points
and the data values.  If the size of the input array is \textless{}1,
then the value \sphinxhyphen{}1 is returned and a warning message is printed
out.

\end{description}

\begin{sphinxadmonition}{note}{Note:}\begin{itemize}
\item {} 
The default names for the variables are \sphinxstyleemphasis{x}, \sphinxstyleemphasis{y}, \sphinxstyleemphasis{z} and \sphinxstyleemphasis{data}.

\item {} 
The data type is automatically detected from the data type of the 
data.

\end{itemize}
\end{sphinxadmonition}

\end{fulllineitems}

\index{power() (in module s2Dcd.gslibnumpy)@\spxentry{power()}\spxextra{in module s2Dcd.gslibnumpy}}

\begin{fulllineitems}
\phantomsection\label{\detokenize{appendices:s2Dcd.gslibnumpy.power}}\pysiglinewithargsret{\sphinxcode{\sphinxupquote{s2Dcd.gslibnumpy.}}\sphinxbfcode{\sphinxupquote{power}}}{\emph{h}, \emph{c=1.0}, \emph{omega=1.5}}{}
Definition of the model of a power variogram.
\begin{description}
\item[{Parameters:}] \leavevmode\begin{description}
\item[{h: float}] \leavevmode
The values of lag \(h\) where to plot the variogram

\item[{c: float}] \leavevmode
The sill value \(c\)

\item[{omega: float}] \leavevmode
The exponent of the model \(\omega\)

\end{description}

\item[{Returns:}] \leavevmode
The values of the variogram \(\gamma(h)\)

\end{description}

\end{fulllineitems}

\index{spherical() (in module s2Dcd.gslibnumpy)@\spxentry{spherical()}\spxextra{in module s2Dcd.gslibnumpy}}

\begin{fulllineitems}
\phantomsection\label{\detokenize{appendices:s2Dcd.gslibnumpy.spherical}}\pysiglinewithargsret{\sphinxcode{\sphinxupquote{s2Dcd.gslibnumpy.}}\sphinxbfcode{\sphinxupquote{spherical}}}{\emph{h}, \emph{c=1.0}, \emph{a=100.0}}{}
Definition of the model of a Spherical variogram.
\begin{description}
\item[{Parameters:}] \leavevmode\begin{description}
\item[{h: float}] \leavevmode
The values of lag \(h\) where to plot the variogram

\item[{c: float}] \leavevmode
The sill value \(c\)

\item[{a: float}] \leavevmode
The actual range \(a\)

\end{description}

\item[{Returns:}] \leavevmode
The values of the variogram \(\gamma(h)\)

\end{description}

\end{fulllineitems}



\section{The \sphinxstyleliteralintitle{\sphinxupquote{s2Dcd.ext}} module}
\label{\detokenize{appendices:module-s2Dcd.ext}}\label{\detokenize{appendices:the-s2dcd-ext-module}}\index{s2Dcd.ext (module)@\spxentry{s2Dcd.ext}\spxextra{module}}\begin{quote}\begin{description}
\item[{license}] \leavevmode
This file is part of s2Dcd.

s2Dcd is free software: you can redistribute it and/or modify
it under the terms of the GNU General Public License as published by
the Free Software Foundation, either version 3 of the License, or
(at your option) any later version.

s2Dcd is distributed in the hope that it will be useful,
but WITHOUT ANY WARRANTY; without even the implied warranty of
MERCHANTABILITY or FITNESS FOR A PARTICULAR PURPOSE.  See the
GNU General Public License for more details.

You should have received a copy of the GNU General Public License
along with s2Dcd.  If not, see \textless{}\sphinxurl{https://www.gnu.org/licenses/}\textgreater{}.

\item[{this\_file}] \leavevmode
\sphinxtitleref{ext.py}

\item[{Purpose}] \leavevmode
A library containing the definition for the extensions for a number
of formats…

\item[{Version}] \leavevmode
\end{description}\end{quote}
\begin{itemize}
\item {} \begin{description}
\item[{0.0.2, 2013\sphinxhyphen{}02\sphinxhyphen{}26 :}] \leavevmode
Added some formats and improved the readibility.

\end{description}

\item {} \begin{description}
\item[{0.0.1, 2012\sphinxhyphen{}05\sphinxhyphen{}15 :}] \leavevmode
The first version.

\end{description}

\end{itemize}


\chapter{Licence}
\label{\detokenize{licence:licence}}\label{\detokenize{licence::doc}}\begin{quote}

\sphinxstyleemphasis{s2Dcd}, a Python library to interact with some Multiple Point
Simulation codes and apply the sequential 2D simulation with
conditioning data approach.
Copyright (C) 2012\sphinxhyphen{}2018  Alessandro Comunian

This program is free software: you can redistribute it and/or modify
it under the terms of the GNU General Public License as published by
the Free Software Foundation, either version 3 of the License, or
(at your option) any later version.

This program is distributed in the hope that it will be useful,
but WITHOUT ANY WARRANTY; without even the implied warranty of
MERCHANTABILITY or FITNESS FOR A PARTICULAR PURPOSE.  See the
GNU General Public License for more details.

You should have received a copy of the GNU General Public License
along with this program.  If not, see \textless{}\sphinxurl{http://www.gnu.org/licenses/}\textgreater{}.

For question, contributions, etc you can send me and
e\sphinxhyphen{}mail: alessandro DOT comunian AT gmail DOT com
\end{quote}


\chapter{Indices and tables}
\label{\detokenize{index:indices-and-tables}}\begin{itemize}
\item {} 
\DUrole{xref,std,std-ref}{genindex}

\item {} 
\DUrole{xref,std,std-ref}{modindex}

\item {} 
\DUrole{xref,std,std-ref}{search}

\end{itemize}

\begin{sphinxthebibliography}{Comunian}
\bibitem[Comunian2012]{purpose:comunian2012}
Comunian, A.; Renard, P. and Straubhaar, J.
\sphinxstyleemphasis{3D multiple\sphinxhyphen{}point statistics simulation using 2D
training images} Computers \& Geosciences, 2012, \sphinxstylestrong{40}, 49\sphinxhyphen{}65.
\sphinxhref{http://dx.doi.org/10.1016/j.cageo.2011.07.009}{doi:10.1016/j.cageo.2011.07.009}
\bibitem[Deutsch1988]{appendices:deutsch1988}
Deutsch, C. V. \& Journel, A.G. (1988) GSLIB:
Geostatistical Software Library and User’s Guide 2nd
edition. Oxford University Press
\end{sphinxthebibliography}


\renewcommand{\indexname}{Python Module Index}
\begin{sphinxtheindex}
\let\bigletter\sphinxstyleindexlettergroup
\bigletter{s}
\item\relax\sphinxstyleindexentry{s2Dcd.deesse}\sphinxstyleindexpageref{appendices:\detokenize{module-s2Dcd.deesse}}
\item\relax\sphinxstyleindexentry{s2Dcd.ext}\sphinxstyleindexpageref{appendices:\detokenize{module-s2Dcd.ext}}
\item\relax\sphinxstyleindexentry{s2Dcd.grid}\sphinxstyleindexpageref{appendices:\detokenize{module-s2Dcd.grid}}
\item\relax\sphinxstyleindexentry{s2Dcd.gslibnumpy}\sphinxstyleindexpageref{appendices:\detokenize{module-s2Dcd.gslibnumpy}}
\item\relax\sphinxstyleindexentry{s2Dcd.s2Dcd}\sphinxstyleindexpageref{appendices:\detokenize{module-s2Dcd.s2Dcd}}
\item\relax\sphinxstyleindexentry{s2Dcd.utili}\sphinxstyleindexpageref{appendices:\detokenize{module-s2Dcd.utili}}
\end{sphinxtheindex}

\renewcommand{\indexname}{Index}
\printindex
\end{document}